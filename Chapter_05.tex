%%%%%%%%%%%%%%%%%%%%%%%%%%%%%%%%%%%%%%%%%%%%%%%%%%%%%%%
%%                 TSE design margins                %%
%%%%%%%%%%%%%%%%%%%%%%%%%%%%%%%%%%%%%%%%%%%%%%%%%%%%%%%
\chapter{Design margin quantification and optimization}
\chaptermark{Design margin quantification and optimization}
\label{ch:TSEcont}
%%%%%%%%%%%%%%%%%%%%%%%%%%%%%%%%%%%%%%%%%%%%%%%%%%%%%%%

This chapter describes a novel design tool for quantifying the level of overdesign in a product via the notion of excess and buffer which constitute the design margins. These definitions are quantified in a multi-dimensional parameter space using rigorous mathematical tools. Design margins determine the product's capacity for absorbing change and therefore its robustness. However, excessive margins result in overdesign and severe operational costs.

We therefore develop a design methodology for strategically allocating design margins without compromising the product's reliability and ability to meet uncertain requirements.

The methodology in this chapter is demonstrated using an application example from the industry for the remanufacturing of a \ac{TRS}. The thermomechanical model described in Section~\ref{sec:thermomech} along with the load case in Section~\ref{subsec:thermalloadcase} is used to define the design variables and changing parameters involved in the remanufacturing design problem. Remanufacturing is performed by \ac{AM} using laser \ac{DED}. The design space considered in this problem is finite since the decision variables are categorical. This kind of problem was chosen to show importance of high level conceptual decisions on the performance of a design as requirements change gradually.

We begin by defining the methodology for obtaining the set-based solutions for arbitrary design and parameter spaces in Section~\ref{sec:TSEmethods}. We then demonstrate the methodology for the remanufacturing of the \ac{TRS} in Section~\ref{sec:TSEcasestudy}. We present the corresponding results in Section~\ref{sec:TSEresults}. We provide some insights and conclusions about the uses and limitations of the developed framework in Section~\ref{sec:TSEcontsummary}.

%============================ METHODOLOGY ============================%
\section{Methodology} \label{sec:TSEmethods}

We consider a product design problem where requirements change several times throughout the product's lifecycle or development process. In this chapter, we will refer to these two time periods as the product cycle for conciseness. Requirements changes occur after a defined period of time (referred to as an epoch) has elapsed.

A decision regarding product redesign is made at the beginning of each epoch depending on a number of factors. The factors driving these decisions include capability, buffer, excess, and reliability. We will formally define these terms.

%---------------------------------------------------------------------%
% Mathematical definitions
\subsection{Relevant design metrics} \label{subsec:designmetrics}

The parameter space in which our design metrics are defined is the set of values assumed by the changing parameters driving the requirements and feasibility criteria. The parameters vector $\mathbf{p} = \left[p_1 ~ p_2 ~ \cdots ~ p_n\right]^{\mathrm{T}}$ is defined in the multi-dimensional parameter space $\mathbf{p}\in\mathbb{R}^n$, where $n$ is the number of changing parameters.

The feasibility criteria are formulated as constraints that the design must satisfy $\mathbf{t} - \mathbf{g}_{f}(\mathbf{p}) \le 0$, where $\mathbf{t}$ is a vector of threshold values that the constraint function $\mathbf{g}_{f}(\mathbf{p})$ must exceed. Unlike requirements, feasibility constraints are fixed throughout the product cycle. Capability is defined as the set of possible values of a design parameter for which feasibility is maintained \cite{Eckert2019}:
%
\begin{equation} \label{eq:capability}
	\textit{C} = \left\{\mathbf{p} \in \mathbb{R}^n~|~ \mathbf{t} - \mathbf{g}_{f}(\mathbf{p}) \le 0\right\}.
\end{equation}

We represent requirements using a joint \acf{PDF} $F_{\mathbf{X}}\left(\mathbf{p}\right)$\cite{Villanueva2014,Pradlwarter2005,Frangopol2003a,Zhu2013a}. Knowing the capability of a design and the corresponding requirement joint \ac{PDF} we can calculate reliability in terms of the probability that the design satisfies the requirement \cite{ForouzandehShahraki2014,Bucher2009}:
%
\begin{equation} \label{eq:reliability}
	\mathbb{P}(\mathbf{p} \in C) = \dfrac{\int\limits_{C\cap R} F_{\mathbf{X}}(\mathbf{p}) d\mathbf{p}}{\int\limits_{R} F_{\mathbf{X}}(\mathbf{p}) d\mathbf{p}}.
\end{equation}
%
$R$ in the denominator is the requirement set defined by the set of parameter values that yield significant probability density values from the joint \ac{PDF} used.

The requirement set $R$ for a uniform \ac{PDF} is given by
%
\begin{equation} \label{eq:uniformpdf}
	F_\mathbf{X}(\mathbf{p})={\begin{cases}{\dfrac {1}{\prod\limits_{j=1}^{n} \left|b_j - a_j\right|}}&\mathrm {for} \ \mathbf{a}\leq \mathbf{p}\leq \mathbf{b},\\[8pt]0&\mathrm {for} \ \mathbf{p}<\mathbf{a}\ \mathrm {or} \ \mathbf{p}>\mathbf{b}\end{cases}},
\end{equation}
%
where $\mathbf{a}$ and $\mathbf{b}$ are the lower and upper bound vectors, respectively. The requirement set $R$ comprises the values of $\mathbf{p}$ that lie within the bounds $\mathbf{a}$ and $\mathbf{b}$:
%
\begin{equation} \label{eq:requirementsetuniform}
	\textit{R} = \left\{\mathbf{p} \in \mathbb{R}^n~|~\mathbf{a}\leq \mathbf{p}\leq \mathbf{b}\right\}.
\end{equation}

The Gaussian joint \ac{PDF} is given by
%
\begin{equation} \label{eq:gaussianpdf}
	F_\mathbf{X}(\mathbf{p})={\frac {\exp \left(-{\frac {1}{2}}({\mathbf {p} }-{\boldsymbol {\mu }})^{\mathrm {T} }{\boldsymbol {\Sigma }}^{-1}({\mathbf {p} }-{\boldsymbol {\mu }})\right)}{\sqrt {(2\pi )^{n}|{\boldsymbol {\Sigma }}|}}},
\end{equation}
%
where $\boldsymbol{\mu}$ is the mean vector and $\boldsymbol{\Sigma}$ is the covariance matrix. In this chapter, we assume that parameters are uncorrelated. This results in a diagonal covariance matrix given by $\boldsymbol{\Sigma} = \mathrm{diag}\left(\boldsymbol{\sigma}\right)$, where $\boldsymbol{\sigma}$ is the standard deviation vector. In the denominator of Equation~(\ref{eq:gaussianpdf}), $|{\boldsymbol {\Sigma }}|\equiv \det {\boldsymbol {\Sigma }} \equiv \prod\limits_{j=1}^{n} \sigma_j$. The requirement set $R$ is defined as the values of $\mathbf{p}$ that result in a probability density level greater than that at the $3 \boldsymbol{\sigma}$ isocontour of a Gaussian $F_\mathbf{X}(\mathbf{p})$.

The requirement set $R$ is defined as the values of $\mathbf{p}$ that result in a probability density level greater than that at the $3{\sigma}$ isocontour of a Gaussian $F_\mathbf{X}(\mathbf{p})$
%
\begin{equation} \label{eq:requirementsetgaussian}
	\textit{R} = \left\{\mathbf{p} \in \mathbb{R}^n~|~F_\mathbf{X}(\mathbf{p}) \geq F_\mathbf{X}(\boldsymbol{\mu} + \left[3\sigma_1~0~\cdots~0\right]^\mathrm{T}) \right\}.
\end{equation}
%
This is because the probability that a random parameter value sampled from a Gaussian \ac{PDF} lies outside the $3\sigma$ isocontour is small ($<0.3\%$). Note that $F_\mathbf{X}(\boldsymbol{\mu} + \left[3\sigma_1~0~\cdots~0\right]^\mathrm{T}) \equiv F_\mathbf{X}(\boldsymbol{\mu} + \left[0~3\sigma_2~\cdots~0\right]^\mathrm{T}) \equiv F_\mathbf{X}(\boldsymbol{\mu} + \left[0~0~\cdots~3\sigma_n\right]^\mathrm{T})$ since they all lie on the $3\sigma$ isocontour.

We use Monte Carlo integration to approximate the integrals in Equation~(\ref{eq:reliability}). Monte Carlo integration based on \acf{LH} sampling has the advantage of scaling well with dimensionality of the problem \cite{Magnusen1997,Zhang2016}, while importance sampling can be used in the case of a Gaussian \ac{PDF} to enhance the accuracy of the approximation \cite{Frangopol2003a,ForouzandehShahraki2014,Kleiber2004}. The Monte Carlo approximation is given by
%
\begin{equation} \label{eq:reliabilitymontecarlo}
	\mathbb{P}(\mathbf{p} \in C) \approx \dfrac{\sum\limits_{i=1}^{|{C\cap R}|} F_{\mathbf{X}}(\mathbf{p}_i)}{\sum\limits_{i=1}^{|R|} F_{\mathbf{X}}(\mathbf{p}_i)}.
\end{equation}

We use a two-dimensional parameter space to illustrate the calculation of the reliability represented by  $\mathbb{P}(\mathbf{p} \in C)$ as shown in Figure~\ref{fig:2Dexamplereliability}.
%
\begin{figure}[h]
	\centering
	\includegraphics[width=0.9\textwidth]{1_2D_example_reliability.pdf}
	\caption{{\color{red} Contours of feasibility constraint $g_{f1}(\mathbf{p})$ in the two-dimensional parameter space for uniform (left) and Gaussian (right) \acp{PDF}}}
	\label{fig:2Dexamplereliability}
\end{figure}
%
Only the Monte Carlo samples that lie within the set $C$ (shown in green) are evaluated by $F_{\mathbf{X}}(\mathbf{p})$ and summed to compute the numerator of Equation~(\ref{eq:reliabilitymontecarlo}). All the Monte Carlo samples shown in Figure~\ref{fig:2Dexamplereliability} are evaluated by $F_{\mathbf{X}}(\mathbf{p})$ and summed to compute the denominator of Equation~(\ref{eq:reliabilitymontecarlo}). %The reliability for the given requirement and capability is then calculated as the ratio of the two sums.

We can now define buffer in the parameter space as the portion of the capability of a design reserved for changes in requirements \cite{Eckert2019}. In other words, the buffer set is defined as the intersection of sets $C$ and $R$
%
\begin{equation} \label{eq:buffer}
	\textit{B} = \left\{\mathbf{p} \in \mathbb{R}^n~|~\mathbf{p} \in \left(C\cap R\right) \right\}.
\end{equation}

Excess is defined as the portion of the parameter space reserved for possible future changes in the requirements \cite{Eckert2019}. This is reflected by the set of parameter values that lie within the capability set $C$ but not within the requirement set (given by its compliment $R'$). Mathematically this is expressed as follows
% 
\begin{equation} \label{eq:excess}
	\textit{E} = \left\{\mathbf{p} \in \mathbb{R}^n~|~\mathbf{p} \in \left(C\cap R'\right) \right\}.
\end{equation}
%
Note that  $B\cup E = C$.

The sets $R$, $E$, and $B$ are shown in Figure~\ref{fig:2Dexampleexcess} for the two-dimensional parameter space example.
%
\begin{figure}[h]
	\centering
	\includegraphics[width=0.9\textwidth]{2_2D_example_excess.pdf}
	\caption{{\color{red} Buffer and excess relative to a feasibility constraint $g_{f1}(\mathbf{p})$ in the two-dimensional parameter space for uniform (left) and Gaussian (right) \acp{PDF}}}
	\label{fig:2Dexampleexcess}
\end{figure}

We are particularly interested in minimizing excess during the product redesign cycle. We estimate excess using the volume of the set $E$ 
%
\begin{equation} \label{eq:excessmontecarlo}
	V_E \approx \dfrac{1}{N} {\sum\limits_{i=1}^{N} H\left(\mathbf{p}_i\right)}, ~\mathrm{where}~ H\left(\mathbf{p}_i\right)={\begin{cases}1&{\text{if }}\mathbf{p}_i\in E\\0&{\text{otherwise}}\end{cases}},
\end{equation}
%
where $N$ is the number of samples from the parameter space used for the integration. We can use the reliability calculation and the volume of the requirement set $R$ to estimate the volume of the set $E$ indirectly. 

The volume of $R$ can be computed analytically for a uniform or Gaussian distribution using the corresponding hyper-rectangle or hyper-ellipsoid, respectively:
%
\begin{equation} \label{eq:Rmontecarlo}
	V_R = {\begin{cases} \prod\limits_{j=1}^{n} \left|b_j - a_j\right| &{\text{if }}F_\mathbf{X}(\mathbf{p})\text{ is uniform}\\\dfrac{\pi^2}{32}\prod\limits_{j=1}^{n} \left|b_j - a_j\right| &{\text{if }}F_\mathbf{X}(\mathbf{p})\text{ is Gaussian}\end{cases}}.
\end{equation}
%
We can then estimate the volume of set $C$ similarly:
%
\begin{equation} \label{eq:Cmontecarlo}
	V_C \approx \dfrac{1}{N} {\sum\limits_{i=1}^{N} H\left(\mathbf{p}_i\right)}, ~\mathrm{where}~ H\left(\mathbf{p}_i\right)={\begin{cases}1&{\text{if }}\mathbf{p}_i\in C\\0&{\text{otherwise}}\end{cases}}.
\end{equation}

The reliability approximates the percentage of $R$ in $C$ and can be used as as proxy for the volume of $C\cap R$ such that $V_{C\cap R} \approx \mathbb{P}(\mathbf{p} \in C) \times V_R$. $V_E$ can now be approximated using
%
\begin{equation} \label{eq:excesssimple}
	V_E \approx V_C - \mathbb{P}(\mathbf{p} \in C) \times V_R.
\end{equation}

Having defined all the required design metrics, we can formulate an optimization problem to minimize excess subject to reliability constraints. We first set the context of the optimization problem in terms of an epoch-era analysis \cite{Ross2008} to simulate changing requirements throughout the product cycle.

%---------------------------------------------------------------------%
% Epoch era analysis
\subsection{Epoch-era analysis for product redesign} \label{subsec:epochera}

We consider the redesign of component as time progresses through its development and lifecycle. At every epoch in the product cycle, the designer must make redesign decisions. The set of redesign choices is defined as the set of non-negative integers $\mathcal{D} = \left\{0,1,2,\cdots,q\right\}$. {\color{red} In practice, the set of redesign choices is derived from a finite number of feasible design alternatives that the designer wishes to include in their decision-making.} Chaining multiple choices together results in a design arc defined as
%
\begin{equation} \label{eq:designarc}
	\mathbf{D} = \left[D_1 ~ D_2 ~ \cdots ~ D_o\right]%^{\mathrm{T}}
\end{equation}
%
with possible choices $D_d \in \mathcal{D}^o$, where $1 \leq o \leq q+1$. The maximum number of redesign choices $q + 1$ dictates the maximum number of possible design arc combinations where no choice is repeated twice. For example, consider a case where there are $q + 1 = 3$ redesign choices given by $\mathcal{D} = \left\{0,1,2\right\}$. If $o=1$ then we have three possible design arcs: $\mathbf{D} = \left[0\right]$, $\mathbf{D} = \left[1\right]$, and $\mathbf{D} = \left[2\right]$. For $o=2$, 6 additional design arcs can be obtained by permuting any 2 choices from $\mathcal{D}$. Similarly, another 6 design arcs can be obtained for $o=3$ by permutating all three choices in set $\mathcal{D}$. A total of 15 design arcs can be obtained from $\mathcal{D} = \left\{0,1,2\right\}$.

These enumerations comprise a set of possible design arcs given by $\Omega_D$. The cardinality of $\Omega_D$ for a different number of redesign choices $q+1$ is given by
%
\begin{equation} \label{eq:cardinality_eq}
	\Omega_D = \sum\limits_{o=1}^{q+1}\Myperm[q+1]{o},
\end{equation}
%
where $\Myperm[q+1]{o}$ is the number of ways for obtaining an ordered subset of $o$ elements from a set of $q+1$ elements.

We define a product cycle with $m$ number of epochs. {\color{red} We assume that the number of epochs is given a priori by the number of design revisions that a product encounters during its cycle}. A decision vector $S \in \mathcal{S}^m$ is defined, where $S_k$ is the decision at epoch $k$ and $\mathcal{S}$ is the set of possible decisions. In this chapter, we consider discrete redesign choices only. A non-negative integer value from the set $\mathcal{S}$ implies a redesign choice. A value of $-1$ implies no redesign is performed at the current epoch. This means that $\mathcal{S} = \left\{-1,0,1,2,\cdots,q\right\}$, where $q + 1$ is the number of available redesign choices.

The vector of all the decisions taken throughout the product cycle is referred to as the decision arc and is defined as
%
\begin{equation} \label{eq:decisionarc}
	\mathbf{S} = \left[S_1 ~ S_2 ~ \cdots ~ S_m\right].
\end{equation}

The set of possible decision arcs that can be generated from the set of possible decisions $\mathcal{S}$ may be restricted by constraints. A feasible decision arc cannot contain repeated choices. For example, the decision arc $\mathbf{S} = \left[0 ~ -1 ~ 0 ~ -1 ~ 1 ~ 3\right]$ is infeasible since the choice $0$ was repeated twice. Furthermore, the first decision cannot be empty, i.e. $S_1 \neq -1$. This is because a decision arc must begin with some sort of design. These restrictions make computing the cardinality of the set of possible decision arcs $\Omega_S$ challenging. However, the cardinality of $\Omega_S$ is given by a finite positive integer similar to $\Omega_D$ in Equation~(\ref{eq:cardinality_eq}).

A corresponding design arc can be extracted from the decision arc by removing all negative elements from $\mathbf{S}$ to obtain $\mathbf{D} = \left[0 ~ 2 ~ 1 ~ 3\right]$. 

For each epoch $k$, a design arc can be extracted by excluding values from $\mathbf{S}$ that are equal to $-1$. The vector $\mathbf{D}_k = \left[D_1 ~ D_2 ~ \cdots ~ D_o\right]$ represents this design arc at epoch $k$ where the elements of $\mathbf{D}$ are non-negative integers and $o$ is the number of non-negative integers in $\mathbf{S}_k = \left[S_1 ~ S_2 ~ \cdots ~ S_k\right]$ up to the current epoch $k$. E.g., for a problem with $m=6$ epochs and the decision arc
%
\begin{equation*} \label{eq:decisionarcex}
	\mathbf{S} = \left[0 ~ -1 ~ 2 ~ -1 ~ 1 ~ 3\right],
\end{equation*}
%
the following $m=6$ design arcs can be extracted 
%
\begin{equation*}
	\begin{aligned}
		& \mathrm{epoch~} k=1: \mathbf{D}_1 = \left[0\right]\\
		& \mathrm{epoch~} k=2: \mathbf{D}_2 = \left[0\right]\\
		& \mathrm{epoch~} k=3: \mathbf{D}_3 = \left[0 ~ 2\right]\\
		& \mathrm{epoch~} k=4: \mathbf{D}_4 = \left[0 ~ 2\right]\\
		& \mathrm{epoch~} k=5: \mathbf{D}_5 = \left[0 ~ 2 ~ 1\right]\\
		& \mathrm{epoch~} k=6: \mathbf{D}_6 = \left[0 ~ 2 ~ 1 ~ 3\right].\\
	\end{aligned}
\end{equation*}
%
It follows that a design arc cannot feature repeated elements due to the uniqueness of the positive decision arc elements. Each design arc has a unique capability set $C_k$.

We now define the requirement arc as a vector of joint \acp{PDF} that has $m$ elements, where each element corresponds to a different epoch $k$ with a requirement joint \ac{PDF} $F_{\mathbf{X}k}(\mathbf{p})$:
%
\begin{equation} \label{eq:requirementarc}
	\mathbf{R} = \left[F_{\mathbf{X}1}(\mathbf{p}) ~ F_{\mathbf{X}2}(\mathbf{p}) ~ \cdots ~ F_{\mathbf{X}m}(\mathbf{p})\right].%^{\mathrm{T}}.
\end{equation}

The reliability at epoch $k$ (quantified by $\mathbb{P}_k(\mathbf{p} \in C_k)$) can be calculated from $C_k$ (derived from $\mathbf{D}_k$) and the requirement joint \ac{PDF} $F_{\mathbf{X}k}(\mathbf{p})$:
%
\begin{equation} \label{eq:reliabilityvector}
	\mathbf{P}(\mathbf{p} \in \mathbf{C}) = \left[\mathbb{P}_1(\mathbf{p} \in C_1) ~ \mathbb{P}_2(\mathbf{p} \in C_2) ~ \cdots ~ \mathbb{P}_m(\mathbf{p} \in C_m)\right],%^{\mathrm{T}},
\end{equation}
%
where $\mathbf{C}$ is the vector of capability sets $\mathbf{C} = \left[C_1 ~ C_2 ~ \cdots ~ C_m\right]$.%^{\mathrm{T}}$.

At each epoch, a reliability threshold $P_k$ is defined to yield a vector of reliability thresholds defined as
%
\begin{equation} \label{eq:reliabilitythvector}
	\mathbf{P}_{th} = \left[P_1 ~ P_2 ~ \cdots ~ P_m\right].%^{\mathrm{T}}.
\end{equation}

The volume of the excess set $V_{Ek}$ at epoch $k$ can be calculated from $C_k$ and the requirement joint \ac{PDF} $F_{\mathbf{X}k}(\mathbf{p})$ using Equations~(\ref{eq:excess}) and (\ref{eq:excesssimple}).
The cumulative excess for a given decision arc can be formulated as
%
\begin{equation} \label{eq:excesscumulative}
	E_c = \sum\limits_{k=1}^{m} V_{Ek}.
\end{equation}

In addition to the decision arc $\mathbf{S}$, a fixed design concept $c_t\in\mathcal{C}$ is defined. The concept type $c_t$ is selected at the very beginning of the epoch-era analysis and does not change throughout epochs. The choice of $c_t$ dictates the list of redesign choices available for the remainder of the product cycle. $\mathcal{C}$ is the set of concept choices whose elements are all non-negative integers. This means that $\mathcal{C} = \left\{0,1,2,\cdots,r\right\}$, where $r + 1$ is the number of available concept choices.

Each concept type $C_t$ has a set of redesign options $\mathcal{D}_t$ with $q+1$ redesign choices attached to it. Accordingly, each concept type $c_t$ will have a set of possible design arcs $\Omega_{Dt}$ and a set of possible decision arcs $\Omega_{St}$. The combination of a concept and design arc $\left\{c,\mathbf{D}\right\}$ will be referred to as a design arc in this chapter for conciseness. Similarly, the combination of a concept and a decision arc $\left\{c,\mathbf{S}\right\}$ will be referred to as a decision arc.

The cardinality of the set of possible design arcs $\Omega_{cD}$ can be obtained by summing up the cardinalities of all sets $\Omega_{Dt}$ for a given set of concept choices $\mathcal{C}$:
%
\begin{equation} \label{eq:cardinailitycdarc}
	\beta = |\Omega_{cD}| = \sum\limits_{t=0}^{r} |\Omega_{Dt}|,
\end{equation}
%
where $|\Omega_{Dt}|$ can be obtained from Equation~(\ref{eq:cardinality_eq}) given the number of redesign choices $q+1$ for each concept. The set of possible design arcs represents the feasible design space, defined as
%
\begin{equation} \label{eq:feasibledesignset}
	\Omega_{cD} = \left\{\left\{c,\mathbf{D}\right\}_1,\left\{c,\mathbf{D}\right\}_2,\cdots,\left\{c,\mathbf{D}\right\}_\beta\right\}.
\end{equation}

We now formulate an optimization problem for choosing the optimal concept $c$ and decision arc $\mathbf{S}$ such that cumulative excess $E_c$ is minimized subject to reliability constraints:
%
\begin{equation}
	\label{eq:TSEoptproblem}
	\begin{aligned}
		& \underset{\{c,\mathbf{S}\}\in\Omega_{cS}}{\text{minimize}}
		& & {f}(c,\mathbf{S};\mathbf{R}) = E_c = \sum\limits_{k=1}^{m} V_{Ek}(c,\mathbf{D}_k;F_{\mathbf{X}k}(\mathbf{p}))\\
		& \text{subject to}
		& & \mathbf{g}(c,\mathbf{S};\mathbf{R}) = \mathbf{P}_{th} - \mathbf{P}(\mathbf{p} \in \mathbf{C}) \le \mathbf{0}, 
	\end{aligned}
\end{equation}
%
where $\Omega_{cS}$ denotes the set of all feasible decision arcs.

We also define a second optimization problem to simulate the effect of accumulating costs that can be incurred throughout a product's lifecycle (since we focus on aerospace applications in this thesis, we consider the cumulative weight of the decision arc to be a proxy of cumulative costs such as fuel \cite{Thomsen2016}):
\begin{equation}
	\label{eq:optproblemweight}
	\begin{aligned}
		& \underset{\{c,\mathbf{S}\}\in\Omega_{cS}}{\text{minimize}}
		& & {f}(c,\mathbf{S}) = W_c = \sum\limits_{k=1}^{m} W_{k}(c,\mathbf{D}_k)\\
		& \text{subject to}
		& & \mathbf{g}(c,\mathbf{S};\mathbf{R}) = \mathbf{P}_{th} - \mathbf{P}(\mathbf{p} \in \mathbf{C}) \le \mathbf{0}.
	\end{aligned}
\end{equation}

{\color{red} Minimizing the cumulative excess is equivalent to minimizing an equally weighted sum of all excesses for all epochs. The designer may substitute the objective functions in Equations~(\ref{eq:optproblemweight}) and (\ref{eq:TSEoptproblem}) with a weighted sum of excess or cost to place emphasis on a particular epoch during the development process. 

The optimization problems are solved for a given requirement arc that is generated by random sampling techniques at the start of the product cycle. If the problem is solved in real-time as requirements are updated progressively, a nonanticipativity constraint can be added to the problem to include requirements up to the current epoch $k$ \cite{Cardin2017}.}

The problems in Equations~(\ref{eq:TSEoptproblem}) and (\ref{eq:optproblemweight}) are solved using the mixed variable optimization variant of the \ac{MADS} algorithm provided by the \texttt{NOMAD} software package \cite{Abramson2009}. This implementation of \ac{MADS} allow users to specify categorical constraints via an extended poll subroutine and is called during the search step \cite{Abramson2004,Abramson2008}.

The solutions of the problems given by Equations~(\ref{eq:TSEoptproblem}) and Equations~(\ref{eq:optproblemweight}) depend on the requirement arc. Requirements are subject to change; therefore, we adopt a set-based design strategy to address such requirement changes.

%---------------------------------------------------------------------%
% Set-based approach
\subsection{Set-based design to mitigate changing requirements} \label{subsec:SBDproblem}

%f: objective functions
%g: constraint functions
%n: dimensionality of parameter space
%i: requirement set sample index 
%j: component index of bounds vectors for Gaussian and uniform pdfs
%m: number of stages
%k: index of stage
%o: number of designs in design arcs
%d: index of design decision in design arc
%q: number of redesign (deposit) choices
%r: number of concept choices
%s: number of SBD requirement arc samples (cardinality)
%w: index of requirement arc sample
%v: number of choices for joint PDF
%t: type of PDF set
%e: number of linear interpolation samples for generating R_v
%\alpha: number of designs for cut-off
%\beta: number of possible concept/design arcs
%\lambda: index of concept/design arcs
%\zeta: number of possible enumerations of decision arcs from design arcs
%\gamma: index of decision arc in enumerations from design arcs

Our set-based design strategy involves sampling requirement arcs $\mathbf{R}$ from the set of possible requirement arcs $\Omega_R = \left\{\mathbf{R}_1,\mathbf{R}_2\cdots,\mathbf{R}_s\right\}$ with $s = |\Omega_R|$. A sample $\mathbf{R}_w$ involves populating the requirement arc with joint \acp{PDF} at each epoch $k$ by selecting a joint \ac{PDF} from the set of possible joint \acp{PDF} $\mathcal{R} = \left\{F_{\mathbf{X}1}(\mathbf{p}),F_{\mathbf{X}2}(\mathbf{p}),\cdots,F_{\mathbf{X}v}(\mathbf{p})\right\}$, where $v$ is the number of joint \ac{PDF} choices. 

Populating the set $\mathcal{R}$ is based on the designer's experience and previous knowledge in requirements. E.g., if requirements are expected to become well-defined over time around a certain value in the parameter space, then a matrix of mean vectors $\mathbf{M} = \left[\boldsymbol{\mu}_1,\boldsymbol{\mu}_2,\cdots,\boldsymbol{\mu}_e\right]^{\textrm{T}}$ and a matrix of standard deviation vectors $\boldsymbol{\Sigma} = \left[\boldsymbol{\sigma}_1,\boldsymbol{\sigma}_2,\cdots,\boldsymbol{\sigma}_e\right]^{\textrm{T}}$ can be obtained by interpolating between the initial and final states for each type of joint \ac{PDF}. An example of this kind of interpolation search strategy is shown in Figure~\ref{fig:2Dexampleinterp} for the same 2D example used in Section~\ref{subsec:designmetrics}.

\begin{figure}[h!]
	\centering
	\includegraphics[width=0.9\textwidth]{3_2D_example_interpolation.pdf}
	\caption{{\color{red} Contours of feasibility constraint $g_{f1}(\mathbf{p})$ in the 2 dimensional parameter space for different types of requirement \acp{PDF}}}
	\label{fig:2Dexampleinterp}
\end{figure}

The set of possible requirement arcs $\Omega_R$ spans every possible combination of the joint \acp{PDF} (their type, mean, and standard deviation) and has a cardinality given by
%
\begin{equation*}
	\begin{aligned}
		& v = e \times e \times |\mathcal{T}|~\mathrm{and}\\
		& s = |\Omega_R| = m^{v},~\mathrm{respectively}.\\
	\end{aligned}
\end{equation*}

Note that a small increase in the number of joint \ac{PDF} types would cause the cardinality of $\Omega_R$ to increase rapidly since interpolation levels would have to increase as well. For practical reasons, only the first few elements of $\Omega_R$ will be used during the set-based design analysis and $s$ will be capped at $10^5$ samples. The rationale behind these choices is explained in Section~\ref{sec:TSEresults}.

The first set-based solution is obtained by solving the optimization problem in Equation~(\ref{eq:TSEoptproblem}) for every requirement arc sample $\mathbf{R}_w$. The corresponding optimal design arc $\mathbf{D}^*$ can be extracted from the optimal decision arc $\mathbf{S}^*$ to obtain the solution $\mathbf{x}^*(\mathbf{R}_w) = \left\{c^*,\mathbf{D}^*\right\}(\mathbf{R}_w)$. This is done in order to compare the overdesign levels across different design arcs rather than different decision arcs.

The set of parametric optimal design arcs with respect to excess is defined as
%
\begin{equation} \label{eq:SBDexcessopt}
	S_E^* = \left\{\mathbf{x}^*(\mathbf{R}_1),\mathbf{x}^*(\mathbf{R}_2)\cdots,\mathbf{x}^*(\mathbf{R}_s)\right\}.
\end{equation}

We track the number of times a specific design arc $\left\{c,\mathbf{D}\right\}_\lambda$ appears as the solution to the parametric optimization problem in $S_E^*$ via a design optimality vector defined as
%
\begin{equation} \label{eq:optimalityvec}
	\mathbf{N}_E = \left[n_{E1} ~ n_{E2} ~ \cdots ~ n_{E\beta}\right],%^{\mathrm{T}},
\end{equation}
%
where $n_{E\lambda}$ is equal to the number of times design arc $\left\{c,\mathbf{D}\right\}_\lambda \in \Omega_{cD}$ is repeated in $S_E^*$. The top $\alpha$ design arcs with the largest $n_{E\lambda}$ values are selected as the set-based solution representing the best performing design arcs in terms of minimizing overdesign:
%
\begin{equation} \label{eq:SBDexcess}
	S_E = \left\{\left\{c,\mathbf{D}\right\}_{E1},\left\{c,\mathbf{D}\right\}_{E2}\cdots,\left\{c,\mathbf{D}\right\}_{E\alpha}\right\}.
\end{equation}
%
The rationale for selecting $\alpha$ will be described in the context of the application example presented in this chapter.

where the subscript $E1,E2,\cdots,E\alpha$ indicates that the corresponding base is ordered by frequency in the set of parametric optimal solutions with respect to excess $S_E^*$.

The same procedure is repeated for the optimization problem in Equation~(\ref{eq:optproblemweight}) to obtain the second set-based solution comprising the best performing design arcs in terms of minimizing weight throughout the product cycle:
%
\begin{equation} \label{eq:SBDweight}
	S_W = \left\{\left\{c,\mathbf{D}\right\}_{W1},\left\{c,\mathbf{D}\right\}_{W2}\cdots,\left\{c,\mathbf{D}\right\}_{W\alpha}\right\}.
\end{equation}

The pseudo-algorithm in Algorithm~\ref{algo:SBDOptalgo} summarizes the above described the methodology for obtaining the sets of optimal design arcs with respect to excess or weight.

\begin{algorithm}
	\DontPrintSemicolon % Some LaTeX compilers require you to use \dontprintsemicolon instead
	\KwIn{
		Set of possible requirement arcs $\Omega_R$, Set of possible design arcs $\Omega_{cD}$
	}
	\KwOut{$S_{E}$}
	{\color{red} Initialize $S_{E}^* = \emptyset$}\;	
	Initialize design optimality vector $\mathbf{N}_E = \left[n_{E1},n_{E2},\cdots,n_{E\beta}\right] = \mathbf{0}$\;	
	\For{$w =1, 2, ..., s$} {
		Solve the parametric optimization problem in Equation~(\ref{eq:TSEoptproblem}) to obtain optimal decision arc $\mathbf{x}_S^*(\mathbf{R}_w) = \left\{c^*,\mathbf{S}^*\right\}(\mathbf{R}_w)$\;
		{\color{red}Obtain optimal design arc from optimal decision arc} by eliminating $-1$ components of $\mathbf{S}^*$ to obtain $\mathbf{x}^*(\mathbf{R}_w) = \left\{c^*,\mathbf{D}^*\right\}(\mathbf{R}_w)$\;
		Augment $S_{E}^* \gets S_{E}^* \cup \{ \mathbf{x}^*(\mathbf{R}_w) \} $\;
		Find the unique index $\lambda$ corresponding to $\left\{c^*,\mathbf{D}^*\right\}$ in $\Omega_{cD}$\;
		Award design arc $n_{E\lambda} \gets n_{E\lambda} + 1$\;
	}
	Sort design optimality vector $\mathbf{N}_E$ in descending order\;
	Select top $\alpha$ design arcs with largest values $n_{E\lambda}$ to obtain set of optimal designs $S_E = \left\{\left\{c,\mathbf{D}\right\}_{E1},\left\{c,\mathbf{D}\right\}_{E2}\cdots,\left\{c,\mathbf{D}\right\}_{E\alpha}\right\}$\;
	\caption{Pseudo-algorithm for obtaining the set of optimal design arcs $S_{E}$}
	\label{algo:SBDOptalgo}
\end{algorithm}

The third set-based solution is the robust design set. We define robustness of a design arc by the number of design arcs satisfied from the set $\Omega_R$. We evaluate the feasibility of each design arc $\left\{c,\mathbf{D}\right\}_\lambda$ sampled from the set of possible design arcs in Equation~(\ref{eq:feasibledesignset}) with respect to every requirement arc $\mathbf{R}_w$ in $\Omega_R$.

We generate all the possible decision arcs for a given design arc $\left\{c,\mathbf{D}\right\}_\lambda$ by randomly inserting the $-1$ decisions into the design arc vector until it has the same number of elements as the number of epochs $m$. We show this using an example. Consider the design arc $\left\{c = 1,\mathbf{D} = \left[2 ~ 0 ~ 1\right]\right\}$. The possible decision arcs are
%
\begin{equation*}
	\begin{aligned}
		& \left\{c = 1,\mathbf{S} = \left[2 ~ -1 ~ -1 ~ -1 ~ 0 ~ 1\right]\right\}\\
		& \left\{c = 1,\mathbf{S} = \left[2 ~ -1 ~ -1 ~ 0 ~ -1 ~ 1\right]\right\}\\
		& \left\{c = 1,\mathbf{S} = \left[2 ~ -1 ~ -1 ~ 0 ~ 1 ~ -1\right]\right\}\\
		& \left\{c = 1,\mathbf{S} = \left[2 ~ -1 ~ 0 ~ -1 ~ 1 ~ -1\right]\right\}\\
		& \left\{c = 1,\mathbf{S} = \left[2 ~ -1 ~ 0 ~ 1 ~ -1 ~ -1\right]\right\}\\
		& \left\{c = 1,\mathbf{S} = \left[2 ~ 0 ~ -1 ~ 1 ~ -1 ~ -1\right]\right\}\\
		& \left\{c = 1,\mathbf{S} = \left[2 ~ 0 ~ 1 ~ -1 ~ -1 ~ -1\right]\right\},\\
	\end{aligned}
\end{equation*}
%
yielding the set of decision arcs with $\zeta = 7$ elements
%
\begin{equation} \label{eq:enumeratedcS}
	S_{cD} = \left\{\left\{c,\mathbf{S}\right\}_{1},\left\{c,\mathbf{S}\right\}_{2}\cdots,\left\{c,\mathbf{S}\right\}_{7}\right\}.
\end{equation}

Feasibility in terms of reliability is checked for every possible decision arc $\left\{c,\mathbf{S}\right\}_{\gamma}$ for a given $\left\{c,\mathbf{D}\right\}_\lambda$ and requirement arc $\mathbf{R}_w$ using $\mathbf{g}(c_{\gamma},\mathbf{S}_{\gamma};\mathbf{R}_w) = \mathbf{P}_{th} - \mathbf{P}(\mathbf{p} \in \mathbf{C}) \le \mathbf{0}$. If any of the decision arcs in set $S_{cD}$ satisfy all the reliability constraints then the corresponding design arc $\left\{c,\mathbf{D}\right\}_\lambda$ is considered feasible. We track the number of requirement arcs satisfied by design arc $\left\{c,\mathbf{D}\right\}_\lambda$ through a robustness vector defined as
%
\begin{equation} \label{eq:robustnessvec}
	\mathbf{N}_R = \left[n_{R1} ~ n_{R2} ~ \cdots ~ n_{R\beta}\right], %^{\mathrm{T}},
\end{equation}
%
where $n_{R\lambda}$ is equal to the number of requirement arcs $\mathbf{R}_w \in \Omega_R$ satisfied by design arc $\left\{c,\mathbf{D}\right\}_\lambda \in \Omega_{cD}$. The top $\alpha$ design arcs with the largest $n_{R\lambda}$ values are considered as the robust design set:
%
\begin{equation} \label{eq:SBDrobust}
	S_R = \left\{\left\{c,\mathbf{D}\right\}_{R1},\left\{c,\mathbf{D}\right\}_{R2}\cdots,\left\{c,\mathbf{D}\right\}_{R\alpha}\right\},
\end{equation}

The final set-based solution is the flexible design set. All possible design arcs in set $\Omega_{cD}$ are ranked in terms of filtered outdegree, defined as the number of possible design arcs that can be obtained from the current design arc by adding exactly one redesign choice that is not an element of the current design arc. The filtered outdegree for a design arc $\left\{c,\mathbf{D}\right\}_\lambda$ having $o$ elements and $q + 1$ redesign choices is equal to
%
\begin{equation} \label{eq:filteredoutdegree}
	O_{F_{\lambda}} = q - o.
\end{equation}

The top $\alpha$ design arcs in terms of filtered outdegree are considered as the flexible design set:
%
\begin{equation} \label{eq:SBDflexible}
	S_F = \left\{\left\{c,\mathbf{D}\right\}_{F1},\left\{c,\mathbf{D}\right\}_{F2}\cdots,\left\{c,\mathbf{D}\right\}_{F\alpha}\right\}.
\end{equation}

The pseudo-algorithm in Algorithm~\ref{algo:SBDRobustalgo} summarizes the above described method for obtaining the sets of robust and flexible design arcs.

\begin{algorithm*}
	\DontPrintSemicolon % Some LaTeX compilers require you to use \dontprintsemicolon instead
	\KwIn{
		Set of possible requirement arcs $\Omega_R$, Set of possible design arcs $\Omega_{cD}$
	}
	\KwOut{$S_{R}$, $S_{F}$}
	Initialize design robustness vector $\mathbf{N}_R = \left[n_{R1},n_{R2},\cdots,n_{R\beta}\right] = \mathbf{0}$\;
	{\color{red} Initialize design flexibility vector $\mathbf{N}_F = \left[O_{F_{1}},O_{F_{2}},\cdots,O_{F_{\beta}}\right] = \mathbf{0}$}\;	
	\For{$\lambda = 1, 2, ..., \beta$} {
		Enumerate possible decision arcs from $\left\{c,\mathbf{D}\right\}_\lambda$ to obtain the set $S_{cD} = \left\{\left\{c,\mathbf{S}\right\}_{1},\left\{c,\mathbf{S}\right\}_{2}\cdots,\left\{c,\mathbf{S}\right\}_{\zeta}\right\}$\;
		\For{$w = 1, 2, ..., s$} {
			\For{$\gamma = 1, 2, ..., \zeta$} {
				
				\If{$\mathbf{g}(c_{\gamma},\mathbf{S}_{\gamma};\mathbf{R}_w) \le \mathbf{0}$} {
					Award design arc $n_{R\lambda} \gets n_{R\lambda} + 1$\;
					\textbf{break}
				}
			
			}
		}
		Compute filtered outdegree for $\left\{c,\mathbf{D}\right\}_\lambda$ using $O_{F_{\lambda}} = q - o$\;
		{\color{red} Update design flexibility vector with $O_{F_{\lambda}}$}\;
	}
	Sort design robustness vector $\mathbf{N}_R$ in descending order\;
	Select top $\alpha$ design arcs with largest values $n_{R\lambda}$ to obtain set of robust design arcs $S_R = \left\{\left\{c,\mathbf{D}\right\}_{R1},\left\{c,\mathbf{D}\right\}_{R2}\cdots,\left\{c,\mathbf{D}\right\}_{R\alpha}\right\}$\;
	Sort design flexibility vector $\mathbf{N}_F$ in descending order\;
	Select top $\alpha$ design arcs with largest values $O_{F_{\lambda}}$ to obtain set of flexible design arcs $S_F = \left\{\left\{c,\mathbf{D}\right\}_{F1},\left\{c,\mathbf{D}\right\}_{F2}\cdots,\left\{c,\mathbf{D}\right\}_{F\alpha}\right\}$\;
	\caption{Pseudo-algorithm for obtaining the sets of robust $S_{R}$ and flexible $S_{F}$ design arcs}
	\label{algo:SBDRobustalgo}
\end{algorithm*}

Figure~\ref{fig:methodology} depicts the flowdiagram of our methodology for obtaining the set-based solutions $S_E,S_W,S_R$, and $S_F$.
%
\begin{figure}[h!]
	\centering
	\includegraphics[width=0.99\textwidth]{4_methodology_V2.pdf}
	\caption{Flowdiagram of methodology for generating set-based solutions}
	\label{fig:methodology}
\end{figure}
%
We will now describe the tradespace used to visualize and compare these solution sets.

%---------------------------------------------------------------------%
% Tradespace exploration
\subsection{Tradespace exploration for comparing solution sets} \label{subsec:TSE}

A tradespace can be constructed by plotting the volume of the capability set $V_c$, against the weight of each design arc. The volume of capability set is chosen as the utility since it is independent of the requirement joint \ac{PDF}. This allows for a fair comparison between different design arcs. The design arcs in sets $S_E,S_W,S_R$, and $S_F$ are also projected on the same tradespace to compare their relative position and size.
The Pareto front for such a tradespace can be approximated by solving the bi-objective problem
%
\begin{equation}
	\label{eq:optproblembiobj}
	\begin{aligned}
		& \underset{\{c,\mathbf{D}\}\in\Omega_{cD}}{\text{minimize}}
		& & \left[ -V_c(c,\mathbf{D}) ~~~W\left(c,\mathbf{D}\right)\right]\\
	\end{aligned}
\end{equation}

The positioning of sets $S_E,S_W,S_R$, and $S_F$ relative to the Pareto set obtained by solving the problem in Equation~(\ref{eq:optproblembiobj}) provides a measure for the dominance of each design set.

%========================= APPLICATION PROBLEM =======================%
\section{Application} \label{sec:TSEcasestudy}

We demonstrate the importance of minimizing excess in aerospace structural component design by applying our method to the design of a \acf{TRS}. The \ac{TRS} is remanufactured using \ac{AM} to increase the stiffness of the outer casing in response to changing requirements (temperature loads). The \ac{TRS} can undergo multiple redesigns as given by a decision arc during its product cycle. We will now describe the available stiffener designs.

%---------------------------------------------------------------------%
% Stiffener and thermomechanical model
\subsection{Stiffener deposition on \ac{TRS} outercasing} \label{subsec:stiffeners}

Stiffeners deposited on the outer casing of the \ac{TRS} involve the application of heat to the outer casing (the substrate) to deposit material on its surface. The application of large heat fluxes to the surface of a structure causes residual distortion that persists after the removal of the heat source. This residual distortion affects the structural performance of the structure when loads are applied during operation. As a result, the residual stresses experienced by the \ac{TRS} due to the deposition of a stiffener must be quantified prior to any structural analysis.

The residual stress tensors that persist after removal of the equivalent heat flux are stored and applied during the analysis of the thermal load case.

There are several stiffener geometries available to the designer of the \ac{TRS} given by the set of possible design arcs $\Omega_{cD}$. We draw inspiration from commonly used standard stiffener designs to generate concepts and design choices \cite{USArmyMaterielCommand1970}. The design space consists of three possible deposition concepts $\mathcal{C} = \left\{0,1,2\right\}$. Concept $c=0$ is a ``wavy" stiffener that has three redesign choices $\mathcal{D}_0 = \left\{0,1,2\right\}$. Concept $c=1$ is a ``hatched" stiffener that has five redesign choices $\mathcal{D}_1 = \left\{0,1,2,3,4\right\}$. Concept $c=2$ is a ``tubular" stiffener that has four redesign choices $\mathcal{D}_2 = \left\{0,1,2,3\right\}$. We illustrate these concepts and their respective design choices in Figure~\ref{fig:designspacestiff}.
%
\begin{figure}[h]
	\centering
	\includegraphics[width=0.45\textwidth]{6_TRS_stiffener_types.pdf}
	\caption{Illustration of possible concepts and redesign choices for \ac{TRS} stiffener}
	\label{fig:designspacestiff}
\end{figure}

We compute the cardinality of the set $\Omega_{cD}$ using Equations~(\ref{eq:cardinality_eq}) and (\ref{eq:cardinailitycdarc}) for this problem as follows:

\begin{equation*}
	\begin{aligned}
		& \mathrm{concept~}c_0:~|\Omega_{D0}| = 15\\
		& \mathrm{concept~}c_1:~|\Omega_{D1}| = 325\\
		& \mathrm{concept~}c_2:~|\Omega_{D2}| = 64\\
		& \beta = |\Omega_{cD}| = 15 + 325 + 64 = 404.\\
	\end{aligned}
\end{equation*}

We now describe the analysis steps for obtaining the capability of a given stiffener design arc as a function of the thermal temperature loads.

%---------------------------------------------------------------------%
% Load case description
\subsection{Loadcase description} \label{subsec:loadcase}

The changing temperature loads in Figure~\ref{fig:thermalloads} are used to specify the vector of changing parameters $\mathbf{p} = \left[ T_1 ~ T_2 ~ T_3 ~ T_4 \right]^{\mathrm{T}}$.

We constrain our study to a parameter space defined by 
%
\begin{equation*}
	\mathbf{p} \in \{ \mathbf{p}: \mathbf{p}_{\textrm{nominal}} -\mathbf{p}_{\textrm{deviation}} \leq \mathbf{p} \leq \mathbf{p}_{\textrm{nominal}} + \mathbf{p}_{\textrm{deviation}} \},
\end{equation*}
%
where $\mathbf{p}_{\textrm{nominal}} =$ $[ 350 ~ 425 ~ 410 ~ 580]^{\mathrm{T}}$ and $\mathbf{p}_{\textrm{deviation}} =$ $[100 ~ 100 ~ 100 ~ 100]^{\mathrm{T}}$. 

In this chapter, we assume that there is no correlation between these temperature loads. This is because environmental factors such as the atmospheric temperature are independent of the exhaust temperature. This allows us to assume that the requirement \acp{PDF} defined in Section~\ref{subsec:designmetrics} feature diagonal covariance matrices.

The \ac{TRS} design is constrained by a safety factor (equal to 2.8) against low-cycle fatigue failure or yielding, whichever occurs first. The thermal load case is cycled and is used to compute the expected fatigue life of the \ac{TRS} using low-cycle fatigue calculations described in Section~\ref{subsec:fatigueanalysis}. 

The structural analysis is performed using a \ac{FE} simulation model which is computationally expensive. As a result, for every design arc in $\Omega_{cD}$, we build a surrogate model for computing $n_{\textrm{safety}}(\mathbf{p})$. The surrogate is built using data obtained from the simulation model for 25 Latin hypercube samples of the parameter space for every design in $\Omega_{cD}$. This resulted in $25\times404=10100$ samples for the surrogate. For this particular problem, the sampling was sufficient to capture the effect of increasing the internal temperature loads ($T_2$,$T_3$, and $T_4$) on decreasing $n_{\textrm{safety}}$ due to the expansion of the outercasing of the \ac{TRS}. We use an open source surrogate model library to build and optimize the hyperparameters of an ensemble of surrogates \cite{Talgorn2018}. 

Mathematically, we formulate the constraint on the safety factor as
%
\begin{equation}
	t_1 - \hat{g}_{f1}(\mathbf{p})\le 0.
\end{equation}
%
We visualize this constraint in the 4-dimensional parameter space for a few example design arcs in Section~\ref{subsec:exampleprob4D}.

%---------------------------------------------------------------------%
% Load case requirements
\subsection{Loadcase requirements} \label{subsec:loadcasereq}

Having defined the 4-dimensional parameter space, we now define the joint \acp{PDF} and the corresponding requirement arcs that can be constructed from them. 

While we consider only two types ($\mathcal{T} = \left\{\textrm{uniform}, \textrm{Gaussian}\right\}$), any distribution can be used in our method.

All design metrics and requirements are scaled between $0$ and $1$. This helps when making comparisons between different design arcs in terms of hypervolume of sets with $0$ being the minimum possible hypervolume and $1$ being the maximum possible hypervolume.

We use $e=5$ interpolation levels to obtain $\mathbf{M}$ and $\boldsymbol{\Sigma}$. The initial mean and standard deviation vectors are $\boldsymbol{\mu}_1 = \left[0.15 ~ 0.80 ~ 0.80 ~ 0.85\right]^{\mathrm{T}}$ and $\boldsymbol{\sigma}_1 = \left[0.1875 ~ 0.125 ~ 0.125 ~ 0.1875\right]^{\mathrm{T}}$, respectively. The final mean and standard deviation vectors are $\boldsymbol{\mu}_5 = \left[0.85 ~ 0.20 ~ 0.20 ~ 0.15\right]^{\mathrm{T}}$ and $\boldsymbol{\sigma}_5 = \left[0.375 ~ 0.250 ~ 0.250 ~ 0.375\right]^{\mathrm{T}}$, respectively. As a result, the matrix of interpolated mean and standard deviation vectors is
%
\begin{equation*}
	\mathbf{M} = \begin{bmatrix}
		\boldsymbol{\mu}_1 ^ {\mathrm{T}} \\[4pt]
		\boldsymbol{\mu}_2 ^ {\mathrm{T}} \\[4pt]
		\boldsymbol{\mu}_3 ^ {\mathrm{T}} \\[4pt]
		\boldsymbol{\mu}_4 ^ {\mathrm{T}} \\[4pt]
		\boldsymbol{\mu}_5 ^ {\mathrm{T}} 
	\end{bmatrix} = \begin{bmatrix}
		0.15 & 0.80 & 0.80 & 0.85 \\[4pt]
		0.325 & 0.65 & 0.65 & 0.675 \\[4pt]
		0.5 & 0.5 & 0.5 & 0.5 \\[4pt]
		0.675 & 0.35 & 0.35 & 0.325 \\[4pt]
		0.85 & 0.20 & 0.20 & 0.15 
	\end{bmatrix}
\end{equation*}
%
and
%
\begin{equation*}
	\boldsymbol{\Sigma} = \begin{bmatrix}
		\boldsymbol{\sigma}_1 ^ {\mathrm{T}} \\[4pt]
		\boldsymbol{\sigma}_2 ^ {\mathrm{T}} \\[4pt]
		\boldsymbol{\sigma}_3 ^ {\mathrm{T}} \\[4pt]
		\boldsymbol{\sigma}_4 ^ {\mathrm{T}} \\[4pt]
		\boldsymbol{\sigma}_5 ^ {\mathrm{T}} 
	\end{bmatrix} = \begin{bmatrix}
		0.1875 & 0.125 & 0.125 & 0.1875 \\[4pt]
		0.234375 & 0.15625 & 0.15625 & 0.234375 \\[4pt]
		0.28125 & 0.1875 & 0.1875 & 0.28125 \\[4pt]
		0.328125 & 0.21875 & 0.21875 & 0.328125 \\[4pt]
		0.375 & 0.250 & 0.250 & 0.375
	\end{bmatrix},
\end{equation*}
%
respectively.

We consider a remanufacturing design problem with $m = 6$ epochs. The number of choices $v$ for $\mathcal{R} = \left\{F_{\mathbf{X}1}(\mathbf{p}),F_{\mathbf{X}2}(\mathbf{p}),\cdots,F_{\mathbf{X}v}(\mathbf{p})\right\}$ and the cardinality $s$ for $\Omega_R$ are
%
\begin{equation*}
	\begin{aligned}
		& v = e \times e \times |\mathcal{T}| = 5 \times 5 \times 2 = 50~~\textrm{and}\\
		& s = |\Omega_R| = m^v = 6^{50},~\mathrm{respectively}.\\
	\end{aligned}
\end{equation*}

Only the first few elements of $\Omega_R$ will be used during the set-based design analysis and $s$ will be capped at $10^5$ samples. This is because the set-based solutions stabilize and do not change after sampling $4 \times 10^4$ requirement arc samples.

While we chose the following reliability threshold vector for this example
%
\begin{equation*}
    \mathbf{P}_{th} = \left[0.01 ~ 0.1 ~ 0.3 ~ 0.3 ~ 0.8 ~ 0.9\right]^{\mathrm{T}},
\end{equation*}
%
a design engineer can test and react to different lifecycle scenarios by adjusting the reliability threshold.

All problem sets, parameters, and constants are summarized in Table~\ref{table:modelsummary}.
%
\begin{table}[h]
	\centering
	\renewcommand{\arraystretch}{1.0}% Wider
	\footnotesize\addtolength{\tabcolsep}{-5pt}
	\caption{Example sets, parameters, and constants}
	\label{table:modelsummary}
	\begin{tabular}{lcc>{\centering\arraybackslash}p{4cm}}
	\hline\hline
	\bf Set/parameter/constant & \bf Notation & \bf Units & \bf Value \\
	\hline
	%================================================================
	Set of concept choices & $\mathcal{C}$ & - & $\left\{0,1,2\right\}$ \\
	Set of deposit choices for concept 0 & $\mathcal{D}_0$ & - & $\left\{0,1,2\right\}$ \\
	Set of deposit choices for concept 1 & $\mathcal{D}_1$ & - & $\left\{0,1,2,3,4\right\}$ \\
	Set of deposit choices for concept 2 & $\mathcal{D}_2$ & - & $\left\{0,1,2,3\right\}$ \\
	Cardinality of design arcs & $\beta$ & - & 404  \\
	Number of epochs & $m$ & - & 6 \\
	Interpolation levels & $e$ & - & 5 \\
	Set of \ac{PDF} types & $\mathcal{T}$ & - & $\left\{\mathrm{``Uniform"},\mathrm{``Gaussian"}\right\}$ \\
	\ac{PDF} initital mean & $\boldsymbol{\mu}_1$ & - & $\left[0.15 ~ 0.80 ~ 0.80 ~ 0.85\right]^{\mathrm{T}}$ \\
	\ac{PDF} initital standard deviation & $\boldsymbol{\sigma}_1$ & - & $\left[0.1875 ~ 0.125 ~ 0.125 ~ 0.1875\right]^{\mathrm{T}}$ \\
	\ac{PDF} final mean & $\boldsymbol{\mu}_e$ & - & $\left[0.85 ~ 0.20 ~ 0.20 ~ 0.15\right]^{\mathrm{T}}$ \\
	\ac{PDF} final standard deviation & $\boldsymbol{\sigma}_e$ & - & $\left[0.375 ~ 0.250 ~ 0.250 ~ 0.375\right]^{\mathrm{T}}$ \\
	Reliability threshold & $\mathbf{P}_{th}$ & - & $\left[0.01 ~ 0.1 ~ 0.3 ~ 0.3 ~ 0.8 ~ 0.9\right]^{\mathrm{T}}$ \\
	Number of \acp{PDF} & $v$ & - & 50 \\
	Number of requirement arc samples & $s$ & - & $10^5$ \\ \hline
	%================================================================
	Nacelle temperature & $T_1$ & $^{o}$C & $300 \pm 100$ \\ 
	Tailcone temperature & $T_2$ & $^{o}$C & $400 \pm 100$ \\ 
	Rotor temperature & $T_3$ & $^{o}$C & $450 \pm 100$ \\ 
	Gas surface temperature & $T_4$ & $^{o}$C & $600 \pm 100$ \\
	%Parameter vector & $\mathbf{p}$ & - & $\left[T_1 ~ T_2 ~ T_3 ~ T_4\right]^{\mathrm{T}}$ \\ 
	\hline
	%================================================================
	Laser power & ${P_\textrm{laser}}$ & W &  3806 \\ 
	Laser beam radius & ${r_l}$ & mm & 14.2 \\ 
	Scanning speed& ${u}$ & mm/s & 5.0 \\ 
	Stiffener height & $S_\textrm{height}$ & mm & 10.0 \\
	Laser penetration depth & $D_p$ & mm & 5.0 \\
	Number of deposition layers & $n_d$ & - & 2 \\
	Threshold low-cycle fatigue safety factor & $t_1$ & - & 2.8 \\
	%================================================================
	\hline\hline
	\end{tabular}
\end{table}

%============================ RESULTS ============================%
\section{Results and discussion} \label{sec:TSEresults}

We initiate the solution of the remanufacturing design problem by obtaining the capability set for every design arc in the set $\Omega_{cD}$. We begin by investigating a few selected design arcs from $\Omega_{cD}$. We then solve a single optimization problem to minimize excess for a given requirement arc from the set $\Omega_R$. We then present the set-based results for the problem using a tradespace.

%---------------------------------------------------------------------%
% Example calculation
\subsection{Example for calculating the design properties of a given design arc} \label{subsec:exampleprob4D}

We use two design arcs from the set $\Omega_{cD}$ to visualize feasible space, capability and reliability in two-dimensional projections of the four-dimensional parameter space in Figure~\ref{fig:4Dexamplepspace}. The design arcs and their filtered outdegree, weight, and hypervolume of capability are reported in Table~\ref{table:pdf4Dexample}.

\newcommand{\cwaa}{0.75cm} % index column width
\newcommand{\cwa}{1.5cm} % design arc column width
\newcommand{\cwc}{1.5cm} % FO column width
\newcommand{\cwd}{1cm} % FO column width
\newcommand{\cwe}{1.5cm} % mean column width
\newcommand{\cwf}{1.5cm} % interval column width
%
%
\newcommand{\cwb}{1.1cm} % Capability column width
\newcommand{\cwi}{1.1cm} % R volume column width
\newcommand{\cwj}{1.1cm} % E volume column width

\begin{table}[h!]
	\centering
	% \renewcommand{\arraystretch}{1.0}% Wider
	\footnotesize\addtolength{\tabcolsep}{-5pt}
	\caption{Results obtained for example design arcs}
	\label{table:pdf4Dexample}
	\begin{tabular}{>{\centering\arraybackslash}p{\cwaa}>{\centering\arraybackslash}p{\cwa}|>{\centering\arraybackslash}p{\cwc}c>{\centering\arraybackslash}p{\cwe}>{\centering\arraybackslash}p{\cwf}cc>{\centering\arraybackslash}p{\cwi}>{\centering\arraybackslash}p{\cwj}>{\centering\arraybackslash}p{\cwb}}
	\hline\hline
	\bf Index & \bf Design arc & \bf Filtered outdegree & \bf Weight &\bf \ac{PDF} mean vector & \bf \ac{PDF} standard deviation vector & \bf \ac{PDF} type & \bf Reliability & \multicolumn{3}{c}{\bf Set volume} \\
	$\lambda$ & $\left\{c,\mathbf{D}\right\}$ & $O_F$ & $W$ &$\boldsymbol{\mu}$ & $\boldsymbol{\sigma}$ & $t$ & $\mathbb{P}\left(\mathbf{p}\in C\right)$ & $V_R$ & $V_E$ & $V_C$ \\ \hline
	%================================================================
	\multirow{5}{\cwaa}{\centering 109} & & \multirow{5}{\cwc}{\centering 2} & \multirow{5}{\cwd}{\centering 13.9 kg} & \multirow{5}{\cwe}{\centering $\begin{bmatrix} 0.375 \\ 0.5 \\ 0.5 \\0.625 \end{bmatrix}$} & \multirow{5}{\cwf}{\centering $\begin{bmatrix} 0.375 \\ 0.125 \\ 0.125 \\0.375 \end{bmatrix}$} & & & & & \multirow{5}{\cwb}{\centering 0.540} \\
	 & $\{c = 1,$ & & & & & ``Uniform" & 0.3089 & 0.0352 & 0.529 \\
	 & $\mathbf{D} = \left[1,2,4\right]\}$ & & & & & & & & & \\
	 & & & & & & ``Gaussian" & 0.165 & 0.0108 & 0.540\\
	 & & & & & & & & & & \\ \hline
	%================================================================
	\multirow{5}{\cwaa}{\centering 110} & & \multirow{5}{\cwc}{\centering 1} & \multirow{5}{\cwd}{\centering 18.5 kg} & \multirow{5}{\cwe}{\centering $\begin{bmatrix} 0.375 \\ 0.5 \\ 0.5 \\0.625 \end{bmatrix}$} & \multirow{5}{\cwf}{\centering $\begin{bmatrix} 0.375 \\ 0.125 \\ 0.125 \\0.375 \end{bmatrix}$} & & & & & \multirow{5}{\cwb}{\centering 0.883} \\
	 & $\{c = 1,$ & & & & & ``Uniform" & 0.759 & 0.0352 & 0.856 \\
	 & $\mathbf{D} = \left[1,2,4,0\right]\}$ & & & & & & & & & \\
	 & & & & & & ``Gaussian" & 0.908 & 0.0108 & 0.875 \\
	 & & & & & & & & & & \\
	%================================================================
	\hline\hline
	\end{tabular}
\end{table}

We can observe that the addition of one more deposit to the design arc $\left\{c = 1, \mathbf{D} = \left[1,2,4\right]\right\}$ increases its performance in terms of capability and reliability. However, this comes at the cost of an additional 4.6 kg of weight, reduced filtered outdegree, and an additional excess (0.327 and 0.335 for uniform and Gaussian \acp{PDF}, respectively). The choice of design arc for a given requirement arc is driven by the need to maintain reliability while minimizing excess. 

{\color{red} The work of \citeauthor{Chen1999} uses capability indices to quantify reliability of a design \cite{Chen1999}. While there are similarities with their approach, there are some notable differences. The calculation of reliability is based on estimating the moments of a normal \ac{PDF} that is assumed to govern the feasibility criteria (referred to as performance) in the presence of changing parameters (referred to as noise variables). Reliability (referred to as capability index) is computed as the distance from the expected value of feasibility to the nearest requirement bound (upper or lower bound) normalized by $3\sigma$. Our approach differs due to its ability to directly relate changing parameters to the feasibility criteria via a response surface. Reliability is then estimated for arbitrary requirement \acp{PDF} using Monte Carlo integration.}

We will apply epoch-era analysis and numerical optimization to solve related design decision problems.

\begin{figure}[h!]
	\centering
	\subfloat[{$\left\{c = 1,\mathbf{D} = \left[1,2,4\right]\right\}$}\label{fig:darc1R1}]{\includegraphics[width=0.9\textwidth]{8a_thermal_out_nominal_RS}}
	
	\subfloat[{$\left\{c = 1,\mathbf{D} = \left[1,2,4,0\right]\right\}$} \label{fig:darc2R1}]{\includegraphics[width=0.9\textwidth]{8b_thermal_out_nominal_RS}}
	\caption{2D projections of isocontours of safety factor in the parameter space}
	\label{fig:4Dexamplepspace}
\end{figure}

%---------------------------------------------------------------------%
% Example calculation
\subsection{Combinatorial optimization with respect to a requirement arc} \label{subsec:exampleoptprob}

We solve the problem given by Equation~(\ref{eq:TSEoptproblem}) using a mixed variable programming version of \ac{MADS} \cite{Abramson2009}. The requirement arc $\mathbf{R}_w$ used for this problem is given in Table~\ref{table:requirementarcex}.

\begin{table}[h!]
	\centering
	\renewcommand{\arraystretch}{1.0}% Wider
	\footnotesize\addtolength{\tabcolsep}{-5pt}
	\caption{Requirement arc $\mathbf{R}_w$}
	\label{table:requirementarcex}
	\begin{tabular}{l>{\centering\arraybackslash}p{4.2cm}>{\centering\arraybackslash}p{6cm}c}
	\hline\hline

	\bf \ac{PDF} Index & \bf \ac{PDF} mean vector & \bf \ac{PDF} interval vector & \bf \ac{PDF} type \\
	$F_\mathbf{X} \in \mathcal{R}$ &$\boldsymbol{\mu}$ & $\boldsymbol{\sigma}$ & $t$ \\ \hline

	\hline
	%================================================================
	$F_{\mathbf{X36}}$ & $\begin{bmatrix} 0.5 & 0.5 & 0.5 & 0.5 \end{bmatrix}$ & $\begin{bmatrix} 0.1875 & 0.125 & 0.125 & 0.1875 \end{bmatrix}$ & "Gaussian" \\
	$F_{\mathbf{X50}}$ & $\begin{bmatrix} 0.85 & 0.2 & 0.2 & 0.15 \end{bmatrix}$ & $\begin{bmatrix} 0.375 & 0.25 & 0.25 & 0.375 \end{bmatrix}$ & "Gaussian" \\
	$F_{\mathbf{X1}}$ & $\begin{bmatrix} 0.15 & 0.8 & 0.8 & 0.85 \end{bmatrix}$ & $\begin{bmatrix} 0.1875 & 0.125 & 0.125 & 0.375 \end{bmatrix}$ & "uniform" \\
	$F_{\mathbf{X46}}$ & $\begin{bmatrix} 0.85 & 0.2 & 0.2 & 0.15 \end{bmatrix}$ & $\begin{bmatrix} 0.1875 & 0.125 & 0.125 & 0.1875 \end{bmatrix}$ & "Gaussian" \\
	$F_{\mathbf{X13}}$ & $\begin{bmatrix} 0.5 & 0.5 & 0.5 & 0.5 \end{bmatrix}$ & $\begin{bmatrix} 0.28125 & 0.1875 & 0.1875 & 0.28125 \end{bmatrix}$ & "uniform" \\
	$F_{\mathbf{X31}}$ & $\begin{bmatrix} 0.325 & 0.65 & 0.65 & 0.675 \end{bmatrix}$ & $\begin{bmatrix} 0.1875 & 0.125 & 0.125 & 0.1875 \end{bmatrix}$ & "Gaussian" \\
	%================================================================
	\hline\hline
	\end{tabular}\\
\end{table}

We plot the results from various decision arcs across epochs in Figure~\ref{fig:epocheraexample}. The first decision arc, $\left\{c=1,\mathbf{S}=\left[2,1,-1,-1,0,-1\right]\right\}$ (shown in red) does not satisfy the reliability constraint as shown in Figure~\ref{fig:Decisionarcreliability}. This is because at epoch $k=3$ the reliability of the corresponding design arc $\left\{c=1,\mathbf{D}=\left[2,1\right]\right\}$ is almost 0. No redesign occurred at epoch $k=3$ when it was needed to increase the reliability of the design arc above the threshold.

We investigate another decision arc $\left\{c=1,\mathbf{S}=\left[4,1,0,2,-1,3\right]\right\}$ (shown in green) that achieves very high reliability throughout all epochs. However, this comes at the cost of increased cumulative excess (green shaded area in Figure \ref{fig:Decisionarcexcess}) relative to that of the first decision arc (red shaded area). % in Figure~\ref{fig:Decisionarcexcess}.

We solve the optimization problem given by Equation~(\ref{eq:TSEoptproblem}) to get the third decision arc $\left\{c=1,\mathbf{S}=\left[4,1,0,2,-1,3\right]\right\}$ (shown in blue) which is optimal in terms of minimizing excess. This decision arc has lower reliability relative to the second decision arc (shown in green) but lower cumulative excess and therefore less overdesign. We provide the values of the objective function and reliability constraints for all three decision arcs in Table~\ref{table:optresultsEA}.

\begin{figure}[h!]
	\centering
	\subfloat[Reliability \label{fig:Decisionarcreliability}]{\includegraphics[width=0.45\textwidth]{10a_stagespace_res}} \hspace{0.1\textwidth}%
	\subfloat[Volume of excess set \label{fig:Decisionarcexcess}]{\includegraphics[width=0.45\textwidth]{10b_stagespace_obj}} \hspace{0.1\textwidth}%	
	\caption{Visualization of decision arcs}
	\label{fig:epocheraexample}
\end{figure}

\newcommand{\ocwa}{0.75cm} % index column width
\newcommand{\ocwb}{4cm} % decision arc column width
\newcommand{\ocwc}{1.5cm} % objective column width
\newcommand{\ocwd}{2cm} % constraints column width
\newcommand{\ocwe}{3cm} % design arc column width

\begin{table}[h!]
	\centering
	% \renewcommand{\arraystretch}{1.0}% Wider
	\footnotesize\addtolength{\tabcolsep}{-5pt}
	\caption{Results obtained for example decision arcs}
	\label{table:optresultsEA}
	\begin{tabular}{>{\centering\arraybackslash}p{\ocwa}>{\centering\arraybackslash}p{\ocwb}|>{\centering\arraybackslash}p{\ocwc}>{\centering\arraybackslash}p{\ocwd}>{\centering\arraybackslash}p{\ocwe}}
	\hline\hline
	\bf Index & \bf Decision arc & \bf Objective value & \bf Reliability constraints & \bf Design arc \\ & $\left\{c,\mathbf{S}\right\}$ & $f(c,\mathbf{S};\mathbf{R})$ & $\mathbf{g}(c,\mathbf{S};\mathbf{R})$ & $\left\{c,\mathbf{D}\right\}$ \\ \hline
	%================================================================
	\multirow{6}{\ocwa}{\centering 1} & & \multirow{6}{\ocwc}{\centering 3.91} & \multirow{6}{\ocwd}{\centering $\begin{bmatrix} -0.063 \\ -0.9 \\ 0.3 \\ -0.7 \\ -0.2 \\ -0.1 \end{bmatrix}$} & \\
	 & & & & \\
	 & $\{c=1,$ & & & $\{c=1,$ \\
	 & $\mathbf{S}=\left[2,1,-1,-1,0,-1\right]\}$ & & & $\mathbf{D}=\left[2,1,0\right]\}$ \\
	 & & & & \\
	 & & & & \\ \hline
	%================================================================
	\multirow{6}{\ocwa}{\centering 2} & & \multirow{6}{\ocwc}{\centering 5.16} & \multirow{6}{\ocwd}{\centering $\begin{bmatrix} -0.94 \\ -0.9 \\ -0.70 \\ -0.7 \\ -0.20 \\ -0.1 \end{bmatrix}$} & \\
	& & & & \\
	& $\{c=1,$ & & & $\{c=1,$ \\
	& $\mathbf{S}=\left[4,1,0,2,-1,3\right]\}$ & & & $\mathbf{D}=\left[4,1,0,2,3\right]\}$ \\
	& & & & \\
	& & & & \\ \hline
	%================================================================
	\multirow{6}{\ocwa}{\centering 3} & & \multirow{6}{\ocwc}{\centering 4.58} & \multirow{6}{\ocwd}{\centering $\begin{bmatrix} -0.063 \\ -0.9 \\ -0.20 \\ -0.7 \\ -0.17 \\ -0.1 \end{bmatrix}$} & \\
	& & & & \\
	& $\{c=1,$ & & & $\{c=1,$ \\
	& $\mathbf{S}=\left[2,1,0,4,-1,3\right]\}$ & & & $\mathbf{D}=\left[2,1,0,4,3\right]\}$ \\
	& & & & \\
	& & & & \\
	%================================================================
	\hline\hline
	\end{tabular}
\end{table}

Finally, the example in this section shows that the order of redesign steps can have a significant impact on the reliability and level of overdesign throughout epochs. The second and third decision arcs contain the same redesign choices but in different order. The differences between them in terms of reliability and cumulative excess reflect the importance of choosing the right order of redesign operations when considering multiple epochs.

We will now solve similar optimization problems for every requirement arc in the set $\Omega_R$ to obtain a set-based solution.

%---------------------------------------------------------------------%
% Example calculation
\subsection{Set-based design and tradespace exploration} \label{subsec:SBDTSE}

We solve an optimization problem similar to the one in Section~\ref{subsec:exampleoptprob} for every requirement arc in $\Omega_R$ to obtain the set of parametric optimal design arcs when optimizing for cumulative excess ($S_E^*$) and cumulative weight ($S_W^*$). We plot the frequency of each design arc in $S_E^*$ and $S_W^*$ and normalize it by the cardinality $\beta$ of set $\Omega_R$ to obtain the histograms shown in Figure~\ref{fig:histogramplotsSBD}. 

We also evaluated the flexibility and robustness of each design arc in $\Omega_{cD}$ using the methodology in Figure~\ref{fig:methodology} and Algorithm~\ref{algo:SBDRobustalgo}. We present the set of design arcs ordered with respect to robustness and flexibility in Figures~\ref{fig:histogramR} and \ref{fig:histogramF}, respectively.

\begin{figure}[h!]
	\centering
	\subfloat[Objective: minimize cumulative excess\label{fig:histogramSBDE}]{\includegraphics[width=0.45\textwidth]{11a_histogram_DOE_E}} \hspace{0.1\textwidth}%
	\subfloat[Objective: minimize cumulative weight\label{fig:histogramSBDW}]{\includegraphics[width=0.45\textwidth]{11b_histogram_DOE_W}} \hspace{0.1\textwidth}%	
	\caption{Distribution of design arcs in optimization driven set-based solutions}
	\label{fig:histogramplotsSBD}
\end{figure}

\begin{figure}[h!]
	\centering
	\subfloat[Set of robust design arcs \label{fig:histogramR}]{\includegraphics[width=0.45\textwidth]{12a_histogram_DOE_R}} \hspace{0.1\textwidth}%	
	\subfloat[Set of flexible design arcs \label{fig:histogramF}]{\includegraphics[width=0.45\textwidth]{12b_histogram_DOE_F}} \hspace{0.1\textwidth}%	
	\caption{Distribution of design arcs in set-based solutions}
	\label{fig:histogramplots}
\end{figure}

We select the top $\alpha = 10$ design arcs in Figure~\ref{fig:histogramplotsSBD} as our set of optimal design arcs for $S_E$ and $S_W$. In practice, $\alpha$ is constrained by the designers' ability to concurrently develop and analyze the selected set of design arcs. For example, development time and cost may limit the designers to a maximum of 10 designs that can be concurrently developed at any given time during development. Furthermore, the 10th design arc in $S_E$ and $S_W$ given by $\lambda = 278$ and $\lambda = 99$, respectively is representative of the lower ranking design arcs since they all have comparable frequencies. 

A similar rationale is used for obtaining the set of robust design arcs $S_R$. Only $\alpha=5$ design arcs are used to construct the flexible set-based solution $S_F$ since we focus on those designs with maximum possible filtered outdegree $O_F = 4$.

The sets $S_E$, $S_W$, $S_R$, and $S_F$ are visualized on a tradespace. This tradespace is described by a utility (given by the volume of the capability set $V_c$) and cost (given by the weight $W$) and is shown in Figure~\ref{fig:tradespaceSBD}. The Pareto front for the tradespace is obtained by solving the problem in Equation~(\ref{eq:optproblembiobj}).

\begin{figure}[h!]
	\centering
	\subfloat[Set-based design arcs for minimizing cumulative excess\label{fig:tradespaceSBDE}]{\includegraphics[width=0.45\textwidth]{13a_tradespace_pareto_E}} \hspace{0.1\textwidth}%
	\subfloat[Set-based design arcs for minimizing cumulative weight\label{fig:tradespaceSBDW}]{\includegraphics[width=0.45\textwidth]{13b_tradespace_pareto_W}} \hspace{0.1\textwidth}%	
	\caption{Tradespace of set-based design arcs}
	\label{fig:tradespaceSBD}
\end{figure}

From the tradespace, we can draw several insights. The flexible set-based solution minimizes cost but also minimizes utility. In contrast, the robust design set maximizes utility but also maximizes the cost. The set-based solution obtained by optimization with respect to excess or weight balances utility with cost. The weight-optimized set-based solution has a larger spread than the excess-optimized solution. We quantify the size of the set-based solutions by their convex hulls. We use the convex hull to calculate three metrics: the area spanned by the set, location of the set given by its centroid and proximity to the Pareto front given by the distance from the centroid to the nearest Pareto point \cite{Brown2019}.
%
We report these convex hull metrics in Table~\ref{table:convexhullresults}.
%
\begin{table*}[h!]
	\centering
	\footnotesize\addtolength{\tabcolsep}{-2pt}
	\caption{Set-based solution comparison}
	\label{table:convexhullresults}
	\begin{tabular}{>{\arraybackslash}p{1.8cm}|C{1cm}C{1cm}|C{1cm}C{1cm}|C{1cm}C{1cm}|C{1cm}C{1cm}|C{1cm}C{1cm}}
		\toprule\toprule
		\multirow{2}{2cm}{\textbf{Quantity}} & \multicolumn{2}{c|}{Set of feasible} & \multicolumn{2}{c|}{Set of robust} & \multicolumn{2}{c|}{Set of flexible} & \multicolumn{2}{c|}{Set of optimal} & \multicolumn{2}{c}{Set of optimal}\\ 
		 & \multicolumn{2}{c|}{design arcs $\Omega_{cD}$} & \multicolumn{2}{c|}{design arcs $S_R$} & \multicolumn{2}{c|}{design arcs $S_F$} & \multicolumn{2}{c|}{design arcs $S_E$} & \multicolumn{2}{c}{design arcs $S_W$}\\ \hline
		%================================================================
		Coordinates & $W$ & $V_c$ & $W$ & $V_c$ & $W$ & $V_c$ & $W$ & $V_c$ & $W$ & $V_c$\\
		\hline
		Lower & 4.32 & 0.059 & 15.66 & 0.992 & 4.32 & 0.127 & 11.47 & 0.736 & 4.32 & 0.369\\
		Upper & 26.74 & 1.00 &  26.74 & 1.00 & 11.28 & 0.812 & 24.70 & 1.00 & 15.66 & 1.00\\
		Set centroid & 14.02 & 0.533 & 22.34 & 0.998 & 6.86 & 0.516 & 16.35 & 0.920 & 11.15 & 0.868 \\ \hline
		%================================================================
		$V_{\textrm{hyper-rectangle}}$ & \multicolumn{2}{c}{1} & \multicolumn{2}{c}{0.0038} & \multicolumn{2}{c}{0.226} & \multicolumn{2}{c}{0.166} & \multicolumn{2}{c}{0.339}\\
		$V_{\textrm{convhull}}$ & \multicolumn{2}{c}{0.758} & \multicolumn{2}{c}{0.0026} & \multicolumn{2}{c}{0.092} & \multicolumn{2}{c}{0.104} & \multicolumn{2}{c}{0.126}\\
		$\%V_{\textrm{feasible}}$ & \multicolumn{2}{c}{75.8\%} & \multicolumn{2}{c}{0.26\%} & \multicolumn{2}{c}{9.2\%} & \multicolumn{2}{c}{10.4\%} & \multicolumn{2}{c}{12.6\%}\\ \hline
		%================================================================
		$D_{\textrm{Pareto}}$ & \multicolumn{2}{c}{0.421} & \multicolumn{2}{c}{0.298} & \multicolumn{2}{c}{0.172} & \multicolumn{2}{c}{0.0904} & \multicolumn{2}{c}{0.0432}\\ 
		%================================================================
		\toprule\toprule
	\end{tabular}
\end{table*}

We can see that the set of optimal design arcs with respect to cumulative excess $S_E$ occupies $10.4\%$ of the objective space which is comparable to that occupied by the set of flexible design arcs $S_F$ and greater than that occupied by the set of robust design arcs $S_R$. The set of optimal design arcs with respect to cumulative weight $S_W$ occupies $12.6\%$ which is comparable to the volume of $S_E$.
%
Furthermore, we can see that the sets of optimal design arcs $S_E$ and $S_W$ are close to the Pareto front, which is expected since these sets aim to balance robustness with flexibility which are indirectly related to capability and weight.

Although there are a lot of commonalities between sets $S_E$ and $S_W$, there are some notable differences. Set $S_E$ favors designs that have higher capability when compared to $S_W$ as given by the $V_c$ coordinate of their centroids of 0.920 and 0.868, respectively. The discrepancy between the two sets is due to the fact that weight, or in more general cases, cost does not necessarily translate to excess. For example two designs of identical weight may have different excesses due to differences in the placement of the stiffener. It is therefore important that the designers carefully select their desired metric for optimization in the problem given by Equation~(\ref{eq:TSEoptproblem}).

We aim to analyze the top 6 designs in sets $S_E$ and $S_W$ in Figure~\ref{fig:histogramplotsSBD} by using the reduced tradespace shown in Figure~\ref{fig:reducedTSE}. We also display the geometry of the deposits that belong to these designs in Tables~\ref{table:depositionsequence_SE} and \ref{table:depositionsequence_SW}.

\begin{figure}[h!]
	\centering
	\subfloat[$S_E$\label{fig:reducedTSE_SE}]{\includegraphics[width=0.5\textwidth]{14a_tradespace_pareto_E_reduced}}
	%
	\subfloat[$S_W$\label{fig:reducedTSE_SW}]{\includegraphics[width=0.5\textwidth]{15a_tradespace_pareto_W_reduced}}
	\caption{Reduced tradespace of solutions in sets $S_E$ and $S_W$}
	\label{fig:reducedTSE}
\end{figure}

Figure~\ref{fig:reducedTSE_SE} and Table~\ref{table:depositionsequence_SE} show that the top 6 designs in $S_E$ share the same concept $c=1$. The top two designs ($\lambda = 82$ and $\lambda = 86$) share the first two deposit choices $D_1=1$ and $D_2=0$. The runner-up design ($\lambda = 86$) adds two additional deposits to the top ranked design ($\lambda = 82$). From Figure~\ref{fig:reducedTSE_SE}, it can be seen that the addition of two more deposits to the top ranked design reduced the capability $V_c$. This is because of the overstiffening of the \ac{TRS} outer casing by the addition of more stiffeners. This overstiffening reduced the fatigue life of the outer casing by inducing concentrated tensile stresses in the unreinforced gaps between the stiffener deposits. The third ranked design ($\lambda = 17$) is identical in geometry to the top ranked designs as seen in Table~\ref{table:depositionsequence_SE} but has the order of the deposit choices interchanged. The discrepancy between $\lambda = 17$ and $\lambda = 82$ is due to the difference in thermomechanical effect of depositing $D_1=0$ first instead of $D_1=1$.

Similar observations can be made for set $S_W$ in Figure~\ref{fig:reducedTSE_SW}. However, the top performing design belonged to concept $c=0$ while the 6th ranked design belonged to concept $c=2$. Furthermore, the number of deposits used for any given design arc did not exceed $o=3$. In contrast, $S_E$ featured many designs with $o=4$. The fewer number of deposits in $S_W$ can be attributed to the preference of the algorithm for minimizing weight above all. Another difference between $S_W$ and $S_E$ is that most designs in $S_W$ are Pareto optimal as given by the proximity metric in Table~\ref{table:convexhullresults}. $S_W$ has a Pareto proximity metric of $0.0432$ units which is smaller than the value of $0.0904$ units belonging to $S_E$. This is due to the reasons explained earlier regarding the discrepancy between weight and excess.  

% We can see that some of the top performing design arcs are sometimes children of a better performing parent design arc. This is the case with design arc $\lambda = 17$ being the parent of $\lambda = 28$ although $\lambda = 17$ is higher up the ranking within $S_E$. 

\renewcommand{\resultsCW}{1.7cm}
\newcommand{\dARA}{\includegraphics[height=1.7cm]{table_E_results/C1D1.pdf}}
\newcommand{\dBRA}{\includegraphics[height=1.7cm]{table_E_results/C1D10.pdf}}

\newcommand{\dARB}{\includegraphics[height=1.7cm]{table_E_results/C1D1.pdf}}
\newcommand{\dBRB}{\includegraphics[height=1.7cm]{table_E_results/C1D10.pdf}}
\newcommand{\dCRB}{\includegraphics[height=1.7cm]{table_E_results/C1D102.pdf}}
\newcommand{\dDRB}{\includegraphics[height=1.7cm]{table_E_results/C1D1024.pdf}}

\newcommand{\dARC}{\includegraphics[height=1.7cm]{table_E_results/C1D0.pdf}}
\newcommand{\dBRC}{\includegraphics[height=1.7cm]{table_E_results/C1D01.pdf}}

\newcommand{\dARD}{\includegraphics[height=1.7cm]{table_E_results/C1D3.pdf}}
\newcommand{\dBRD}{\includegraphics[height=1.7cm]{table_E_results/C1D31.pdf}}
\newcommand{\dCRD}{\includegraphics[height=1.7cm]{table_E_results/C1D314.pdf}}
\newcommand{\dDRD}{\includegraphics[height=1.7cm]{table_E_results/C1D3140.pdf}}

\newcommand{\dARE}{\includegraphics[height=1.7cm]{table_E_results/C1D2.pdf}}
\newcommand{\dBRE}{\includegraphics[height=1.7cm]{table_E_results/C1D21.pdf}}
\newcommand{\dCRE}{\includegraphics[height=1.7cm]{table_E_results/C1D210.pdf}}
\newcommand{\dDRE}{\includegraphics[height=1.7cm]{table_E_results/C1D2104.pdf}}

\newcommand{\dARF}{\includegraphics[height=1.7cm]{table_E_results/C1D2.pdf}}
\newcommand{\dBRF}{\includegraphics[height=1.7cm]{table_E_results/C1D21.pdf}}
\newcommand{\dCRF}{\includegraphics[height=1.7cm]{table_E_results/C1D210.pdf}}

\begin{table}[h!]
	\centering
	\renewcommand{\arraystretch}{1.0}% Wider
	\footnotesize\addtolength{\tabcolsep}{-5pt}
	\caption{Top performing design arcs in $S_E$}
	\label{table:depositionsequence_SE}
	\begin{tabular}{>{\centering\arraybackslash}m{\resultsCW}>{\centering\arraybackslash}m{\resultsCW}>{\centering\arraybackslash}m{\resultsCW}>{\centering\arraybackslash}m{\resultsCW}>{\centering\arraybackslash}m{\resultsCW}>{\centering\arraybackslash}m{\resultsCW}}
		\hline\hline
	
		\bf Design arc Index & \bf concept & \bf 1st deposit & \bf 2nd deposit & \bf 3rd deposit & \bf 4th deposit \\
		$\lambda$ & $c$ & $D_1$ & $D_2$ & $D_3$ & $D_4$ \\ \hline
		%================================================================
		82 & 1 & \dARA & \dBRA & & \\ 
		86 & 1 & \dARB & \dBRB & \dCRB & \dDRB \\
		17 & 1 & \dARC & \dBRC & & \\ 
		240 & 1 & \dARD & \dBRD & \dCRD & \dDRD \\ 
		167 & 1 & \dARE & \dBRE & \dCRE & \dDRE \\
		164 & 1 & \dARF & \dBRF & \dCRF & \\
		%================================================================
	\hline\hline
	\end{tabular}
\end{table}

\renewcommand{\resultsCW}{1.7cm}
\renewcommand{\dARA}{\includegraphics[height=1.7cm]{table_W_results/C0D2.pdf}}

\renewcommand{\dARB}{\includegraphics[height=1.7cm]{table_W_results/C1D1.pdf}}
\renewcommand{\dBRB}{\includegraphics[height=1.7cm]{table_W_results/C1D10.pdf}}

\renewcommand{\dARC}{\includegraphics[height=1.7cm]{table_W_results/C1D4.pdf}}
\renewcommand{\dBRC}{\includegraphics[height=1.7cm]{table_W_results/C1D41.pdf}}
\newcommand{\dCRC}{\includegraphics[height=1.7cm]{table_W_results/C1D410.pdf}}

\renewcommand{\dARD}{\includegraphics[height=1.7cm]{table_W_results/C1D1.pdf}}
\renewcommand{\dBRD}{\includegraphics[height=1.7cm]{table_W_results/C1D10.pdf}}
\renewcommand{\dCRD}{\includegraphics[height=1.7cm]{table_W_results/C1D104.pdf}}

\renewcommand{\dARE}{\includegraphics[height=1.7cm]{table_W_results/C1D2.pdf}}
\renewcommand{\dBRE}{\includegraphics[height=1.7cm]{table_W_results/C1D21.pdf}}

\renewcommand{\dARF}{\includegraphics[height=1.7cm]{table_W_results/C2D0.pdf}}
\renewcommand{\dBRF}{\includegraphics[height=1.7cm]{table_W_results/C2D03.pdf}}

\begin{table}[h!]
	\centering
	\renewcommand{\arraystretch}{1.0}% Wider
	\footnotesize\addtolength{\tabcolsep}{-5pt}
	\caption{Top performing design arcs in $S_W$}
	\label{table:depositionsequence_SW}
	\begin{tabular}{>{\centering\arraybackslash}m{\resultsCW}>{\centering\arraybackslash}m{\resultsCW}>{\centering\arraybackslash}m{\resultsCW}>{\centering\arraybackslash}m{\resultsCW}>{\centering\arraybackslash}m{\resultsCW}}
		\hline\hline
	
		\bf Design arc Index & \bf concept & \bf 1st deposit & \bf 2nd deposit & \bf 3rd deposit \\
		$\lambda$ & $c$ & $D_1$ & $D_2$ & $D_3$ \\ \hline
		%================================================================
		11 & 0 & \dARA & & \\ 
		82 & 1 & \dARB & \dBRB & \\
		294 & 1 & \dARC & \dBRC & \dCRC \\ 
		93 & 1 & \dARD & \dBRD & \dCRD \\ 
		14 & 1 & \dARE & \dBRE & \\
		352 & 2 & \dARF & \dBRF & \\
		%================================================================
		\hline\hline
		\end{tabular}
\end{table}

\begin{figure}[h!]
	\centering
	\subfloat[$S_E$\label{fig:piechart_concept_SE}]{\includegraphics[width=0.5\textwidth]{14c_concept_pie_chart_E}} 
	%
	\subfloat[$S_W$\label{fig:piechart_concept_SW}]{\includegraphics[width=0.5\textwidth]{15c_concept_pie_chart_W}} 
	\caption{Distribution of concepts in sets $S_E$ and $S_W$}
	\label{fig:piechart_concept}
\end{figure}

We analyze the the distribution of the concept choices $c$ within the set $S_E$ in Figure \ref{fig:piechart_concept_SE}. We can see that concept choice $c=1$ dwarfs the other two concept choices. This is because the ``hatched" concept has deposit choices with relatively high capability and excess. The discontinuous stiffener design featured by its various deposits provided sufficient reinforcement without overstiffening the outer casing. This contrasts the deposit choices available to the "wavy" and "tubular" stiffener concepts which are continuous.

We also look at the distribution of concept choices $c$ within the set $S_W$ in Figure~\ref{fig:piechart_concept_SW}. The concept choices $c=0$ and $c=1$ are comparable in terms of frequency within the set. This is because both concepts contain deposit choices that are low in weight. Since the optimization objective for set $S_W$ is weight, the low-weight deposit choices within these concepts are equally preferred.

\begin{figure}[h!]
	\centering
	\subfloat[$S_E$\label{fig:piechart_D1_SE}]{\includegraphics[width=0.5\textwidth]{14b_1st_stage_pie_E}} 
	%
	\subfloat[$S_W$\label{fig:piechart_D1_SW}]{\includegraphics[width=0.5\textwidth]{15b_1st_stage_pie_W}} 
	\caption{Distribution of first deposit $D_1$ when concept $c=1$ is selected in sets $S_E$ and $S_W$}
	\label{fig:piechart_D1}
\end{figure}

We then analyze the distribution of the first choice $D_1$ in the design arcs within the set $S_E$. This distribution is shown in Figure~\ref{fig:piechart_D1_SE}. We can see from this distribution that the redesign choice $D_1 = 1$ is chosen most frequently due the high capability $V_c$ of this deposit choice and its children. The third most common design choice is $D_1 = 0$, which is comparable in frequency to $D_1 = 2$. However, it should be noted that committing to $D_1 = 0$ restricts the designers' choices later in the product's cycle. Figure~\ref{fig:reducedTSE_SE} shows that $\lambda = 17$ is the only possible design arc with $D_1 = 0$. This is in contrast to $D_1 = 1$ that can be used to obtain two different design arcs $\lambda = 82$ and $\lambda = 86$. These additional insights obtained by close examination of the tradespace can be beneficial as opposed to picking the most obvious result from Figure~\ref{fig:piechart_D1}.

The tools developed in this chapter can help designers make informed decisions about the sequence of redesign choices during a product cycle. These insights are particularly useful for determining the first redesign choice when uncertainty  and the number of possible choices are high.

{\color{red}
%========================= COMPUTATIONAL COST ==========================%
\section{Comment on computational cost of methods used}
\label{sec:comptcostSBD}

As explained in Section~\ref{subsec:loadcase} a surrogate model for $n_{\textrm{safety}}(\mathbf{p})$ is trained from $10100$ samples to estimate capability, reliability, and excess in the parameter space for feasible design arcs. Algorithms~\ref{algo:SBDOptalgo} and \ref{algo:SBDRobustalgo} were used with lookup tables that return the excess and reliability for every design arc $\left\{c,\mathbf{D}\right\} \in \Omega_{CD}$ and requirement \ac{PDF} $F_{\mathbf{X}} \in \mathcal{R}$ to avoid recomputing them during parametric optimization. This results in $404 \times 50 = 20,200$ excess and reliability values that need to be computed using Monte Carlo integration before exercising the algorithms in this chapter. Table~\ref{table:TSEsurrcomputation} shows a breakdown of the computational cost for obtaining the lookup table.
}

\newcommand{\comresultsTSECW}{1.7cm}
\begin{table*}[t]
	\centering
	\renewcommand{\arraystretch}{1.2}% Wider
	\footnotesize\addtolength{\tabcolsep}{-5pt}
	\caption{Breakdown of total computational time for obtaining surrogate models}
	\label{table:TSEsurrcomputation}
	\begin{tabular}{>{\centering\arraybackslash}m{3.5cm}>{\centering\arraybackslash}m{\comresultsTSECW}>{\centering\arraybackslash}m{\comresultsTSECW}>{\centering\arraybackslash}m{\comresultsTSECW}>{\centering\arraybackslash}m{\comresultsTSECW}>{\centering\arraybackslash}m{\comresultsTSECW}}
	\hline\hline
	\bf Task & \bf function evaluations & \bf evaluation time & \bf iterations & \bf iteration time & \bf total time \\ \hline
	%================================================================
	Surrogate $n_{\textrm{safety}}(\mathbf{p})$ & 10100 & 10 min & - & - & 70.1 days \\ \hline
	%================================================================
	Lookup table & $6.06\times10^8$ & 1 ms & 20200 & 30 s & \multirow{4}{\resultsCW}{\centering 7.01 days} \\
	Compute $V_C$ & 10000 & 1 ms & 1 & 10 s & \\
	Compute $V_E$ & 10000 & 1 ms & 1 & 10 s & \\
	Compute $\mathbb{P}(\mathbf{p} \in C)$ & 10000 & 1 ms & 1 & 10 s & \\
	%================================================================
	\bf Total time & \multicolumn{4}{c}{77.1 days} \\
	\hline\hline
	\end{tabular} 
\end{table*}

{\color{red}
A breakdown of the computational cost of Algorithms~\ref{algo:SBDOptalgo} and \ref{algo:SBDRobustalgo} is shown in Table~\ref{table:SBDTSEcomputation}. It can be concluded that obtaining the surrogate model for the feasibility criterion $n_{\textrm{safety}}(\mathbf{p})$ incurs the most computational cost of 70.1 days as opposed to 9.9 days for excersing the algorithms in this chapter. The total computational cost incurred for obtaining solutions to the example in this chapter is 87 days. This is significantly greater than the cost incurred by finding scalable solutions to the problem in Chapter~\ref{ch:scalableSBD} (40.7 days). However, the methods in this chapter provide solutions for multiple epochs as opposed to a single design revision. This means that the methods presented could potentially provide more added value to the component through multiple life extensions.}

\renewcommand{\resultsCW}{1.7cm}
\begin{table*}[t]
	\centering
	\renewcommand{\arraystretch}{1.2}% Wider
	\footnotesize\addtolength{\tabcolsep}{-5pt}
	\caption{Breakdown of total computational time for Chapter~\ref{ch:TSEcont} algorithms}
	\label{table:SBDTSEcomputation}
	\begin{tabular}{>{\centering\arraybackslash}m{\resultsCW}>{\centering\arraybackslash}m{\resultsCW}>{\centering\arraybackslash}m{\resultsCW}>{\centering\arraybackslash}m{\resultsCW}>{\centering\arraybackslash}m{\resultsCW}>{\centering\arraybackslash}m{\resultsCW}>{\centering\arraybackslash}m{\resultsCW}}
	\hline\hline
	\bf Algorithm & \bf Step & \bf function evaluations & \bf evaluation time & \bf iterations & \bf iteration time & \bf total time \\ \hline
	%================================================================
	\multirow{3}{\resultsCW}{\centering \ref{algo:SBDOptalgo}} & step 3 & - & - & 100000 & 5 s & \multirow{2}{\resultsCW}{\centering 2.89 days} \\
	 & step 4 & $\approx$500 & 5 ms & 1 & 2.5 s \\ \hline
	%================================================================
	\multirow{2}{\resultsCW}{\centering \ref{algo:SBDRobustalgo}} & step 3 & - & - & 404 & 25 min & \multirow{2}{\resultsCW}{\centering 7.01 days} \\
	& step 4-9 & $\approx 3\times10^5$ & 5 ms & 1 & 25 min \\ \hline
	%================================================================
	& \bf Total time & \multicolumn{5}{c}{9.90 days} \\
	\hline\hline
	\end{tabular}
\end{table*}

%============================== SUMMARY ================================%
\section{Summary}
\label{sec:TSEcontsummary}

We presented a set-based design tool for generating optimal redesign sequences for a product as its lifecycle or development process progresses. Design arcs are obtained by solving an optimization problem to minimize excess subject to reliability constraints that depend on changing requirements.

We developed a tool for quantifying the level of excess in a design when multiple changing parameters are considered simultaneously. The areas of the multi-dimensional parameter space related to design capability, requirements, excess, or buffer are identified by means of sets, and their volume is obtained using Monte Carlo integration.

By minimizing the volume of the excess set for a particular design and requirement, overdesign costs are mitigated. We model changing parameters by means of requirement arcs. The optimization problem is solved for each requirement arc to obtain a set of corresponding design arcs.

Tradespace exploration is then used to visualize the set of optimal design arcs and compare it to sets of flexible and robust design arcs. It was shown that our approach results in a set of design arcs that balance flexibility with robustness. An examination of the frequency of each design choice within the set of optimal design arcs provides designers with useful insights about the best approach at the early stages of the product's lifecycle or development process when uncertainty is high and decisions have a lasting impact on the product's performance for the remainder of its lifecycle.

The work in this chapter extends the approach used in Chapter~\ref{ch:scalableSBD} to include problems where multiple changes in requirements may occur. This is represented by the requirement arcs discussed in this chapter. Furthermore, robustness in addition to flexibility was quantified and used as a design metric in this chapter.

In the next chapter, we discuss the use of stochastic optimization algorithms for obtaining sets of design solutions when changing parameters govern the design optimization problem. Such approaches can help address the need for extensive exploration of the parameter space or the set of possible requirements (given by $\Omega_R$) by reducing the number of needed function evaluations. We show how such approaches can substitute parts of the algorithms that have been developed in this chapter and Chapter~\ref{ch:scalableSBD}.