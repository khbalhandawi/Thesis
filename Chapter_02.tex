%%%%%%%%%%%%%%%%%%%%%%%%%%%%%%%%%%%%%%%%%%%%%%%%%%%%%%%
%%                    Background                     %%
%%%%%%%%%%%%%%%%%%%%%%%%%%%%%%%%%%%%%%%%%%%%%%%%%%%%%%%
\chapter{Background}
\chaptermark{Background}
\label{ch:background}
%%%%%%%%%%%%%%%%%%%%%%%%%%%%%%%%%%%%%%%%%%%%%%%%%%%%%%%

In this chapter, Section~\ref{sec:remanufacturingdesign} reviews remanufacturing design problems to identify the key enablers of this product recovery route. Section~\ref{sec:changeability} introduces changeable design principles which are necessary to enable remanufacturing design. Section~\ref{sec:margins} introduces metrics for quantifying the level of overdesign in a product which is impacted by the amount of embedded changeability in a product. Section~\ref{sec:SBD} discusses advances in \ac{SBD} for providing sets of design solutions to leverage the added product changeability. Section~\ref{sec:tradespace} introduces a useful design space exploration tool for visualizing and comparing sets of design solutions throughout a product's lifecycle or development process.

%===================== DESIGN FOR REMANUFACTURING ======================%
\section{Product design for remanufacturing} \label{sec:remanufacturingdesign}

The effectiveness of \ac{AM} for remanufacturing \ac{EOL} components is reported by \citeauthor{VanThao2015} and \citeauthor{Wilson2014} \cite{VanThao2015,Wilson2014}. They consider replacement strategies and \ac{EOL} decisions regarding reuse, recycling or remanufacturing. However, there are some notable studies that have introduced remanufacturing considerations into component design.

Level set topology optimization was used by \citeauthor{Liu2017} to optimize a structural component considering subtractive remanufacturing \cite{Liu2017}. A containment constraint is formulated and used to ensure that a remanufactured design is contained within the material domain of the parent design. This methodology yields designs that can be scaled down by remanufacturing. However, it does not consider the reverse operation of remanufacturing by additive methods. Furthermore, variable loading requirements are not considered.

Environmental impact was considered as an optimization objective for a topology optimization problem of a structural component \cite{Tang2016}. Additive manufacturing was accommodated by incorporating \ac{LCA} considerations into the design problem. Although this is not a remanufacturing study, the ability of \ac{AM} to enable remanufacturing is underlined.

An important feature of a product's lifecycle is upgrade, defined as an improvement at the specifications level \cite{Xing2007}. The upgrade levels for remanufacturing of a product are usually predetermined and are not adjusted based on required specifications at the \ac{EOL}. Based on this, a strategy for determining the optimal market position in terms of pricing and remanufacturing costs can be developed \cite{Kwak2013}. \citeauthor{Kwak2013} address the major activities of remanufacturing which include product takeback (the process of collecting \ac{EOL} products for the activity of remanufacturing, modelled using several scenarios where the remanufacturer either passively accepts all \ac{EOL} products or selectively purchases them), remanufacturing operations, and remarketing \cite{Kwak2013}. Decisions are then made regarding the reusability of the \ac{EOL} product's components. The target specifications for components in need of an upgrade are optimized to maximize revenue from resale of the remanufactured product. The upgrade levels for remanufacturing are captured using generational differences defined as the amount of discrepancy between the current component's specifications and those of components in recent cutting edge products.

The previous study describes the importance of designing remanufactured products for variable markets and requirements. There are additional sources of uncertainty in remanufacturing design problems due to the condition of the recovered product prior to remanufacturing and the specification levels its components should meet in order to function within the product system.

Prior research suggests that successful remanufacturing requires a product's components to be designed for variable requirements to maximize environmental benefits. The main principle governing the ease of upgrading component specifications involves design changeability \cite{Suh2007}. As a result, a review of changeable design practices is warranted.

%============================ CHANGEABILITY ============================%
\section{Quantifying changeability in product design}
\label{sec:changeability}

Design changeability is defined as the ability of a system to undergo specified classes of changes with relative ease and efficiency. A design change is effected when the cost of the change is below a specified threshold. This definition was used by \citeauthor{Lawand2019} to make decisions regarding different end-of-life scenarios  \cite{Lawand2019}. However, cost is not the only factor that governs the changeability of a component.

Design changeability addresses the challenges of modern product development which include dynamic marketplaces, rapid technological evolution, and changing operating environments. Design flexibility and robustness are two aspects of design changeability that directly address these challenges.

A product and its operating environment undergo change during design and operation in order to stay relevant in dynamic markets. Change events are characterized by three elements: i) the agent of change, ii) the mechanism of change, and iii) the effect of change. 

The change agent is the instigator for change in the product and is specified in the form of product requirements. The nature of the change agent helps identify the type of change the system must undergo. If the change is external to the product system (e.g., environmental operational conditions) then the change is of a {flexible} type. If the change agent is internal to the system (e.g., sizing and tolerance requirements) then it is of an \textit{adaptable} type. 

The change effect is the difference between the states of a product before and after the change. Based on the nature of change effects, three more changeability aspects are defined. {Robustness} is defined as the insensitivity of the design to internal or external change (e.g., stability of a vehicle despite changes in road conditions and grade). {Scalability} is the ability of the design to change to meet a different level of a specification (e.g., reinforce a structural member to carry a larger load than originally intended). \textit{Modifiability} is the ability of the design to change in order to accommodate unforeseen requirements not native to the original design (e.g., ability of a cargo plane to be repurposed for reconnaissance missions) \cite{Ross2008}. This term is also referred to as \textit{evolvability} in the literature \cite{Tackett2014}. 

A system may undergo some or all types of change. Several works in the literature have attempted to quantify and capture the changeability of a product system for embedding this principle in product design. 

\citeauthor{Tackett2014} use the product system's capability of meeting design requirements to quantify the available excess capacity for evolving \cite{Tackett2014}. Based on the excess available in a product, an evolvability metric based on the principle of stored elastic energy in a system is computed. The evolvability metric is a relative metric that is useful for comparative design studies.

Other studies focus on quantifying flexibility as a result of predictable and unpredictable changes in the operating environment \cite{Olewnik2004,Liu2008}. In one study, the tradeoff between various requirements (referred to as design objectives and performances) is captured by a Pareto set. Movement along the shortest path from one end of the Pareto set to the other is penalized by a change cost. Flexible designs are identified as a ranged set between the extremes of the Pareto set such that the overall change costs are minimized \cite{Olewnik2004}.

The notion of flexible ranged sets is also investigated by other researchers \cite{Liu2008}. Candidate target sets of solutions that maximize a flexibility metric over the set are identified in the design space by mapping flexible designs identified in the requirements space. The design and requirements spaces are defined as the set of possible values the design variables or requirements can assume respectively. The process begins by producing a number of design alternatives through probing the design space. The design alternatives are mapped on the requirements space (referred to as the attribute space). Design alternatives are partitioned into ranged sets in the requirements space. A flexibility metric for each set is calculated by integration of an influence function over the set. Sets that maximize flexibility are preferred as possible design solutions.

\citeauthor{Suh2007} considered {modularity} of product platforms as a means for achieving changeability \cite{Suh2007}. Requirement bandwidths (referred to as product attributes) are computed based on the market conditions for the product platform. Optimization is used to position product platforms in the market (similar to \citeauthor{Kwak2013} \cite{Kwak2013}) and compute design bandwidths. Monte Carlo simulation is used to evaluate the effect of uncertainty in the market on the net present value of the product platform. The sensitivity of flexible and inflexible product platforms to uncertainty is compared via the expected net present value. In this study, only predetermined product variants are considered as part of the product platform. As explained earlier, in a remanufacturing context it is important to adjust the upgrade levels of the product based on changes in the requirements \cite{Kwak2013}.

A quantification of flexibility is shown to be the filtered outdegree of a design within a networked tradespace \cite{Ross2008}. A tradespace is a design exploration tool that assesses the tradeoff between utility of a given design and its associated costs. The utility and cost functions are defined based on the designer's preferences and experience.

In addition to defining metrics for design capability and capacity, \citeauthor{Rehn2018} use the filtered outdegree to quantify the flexibility of enumerated designs in the tradespace \cite{Rehn2018}. The tradeoff between a multi-attribute utility function containing capability, capacity, operability, and flexibility and the acquisition cost is quantified by generating Pareto-optimal designs. The change path taken between several design instantiations is referred to as an arc and governs the rules regarding changeability between different designs  \cite{Rehn2018, Viscito2009, Ross2008,Rapp2018}. \citeauthor{Rehn2018} count the number of end-states when quantifying the filtered outdegree \cite{Rehn2018}. %This is the simplest approach for doing so.

More advanced representations of the filtered outdegree are defined and used in the literature. \citeauthor{Viscito2009} use the value weighted filtered outdegree as a proxy for quantifying flexibility \cite{Viscito2009}. This metric captures the utility difference between an originating design and its possible destination designs such that the best flexible designs that generate an increase in utility during transition are considered during tradespace exploration.

In other studies, flexibility is defined as the ability of a system to be modified to do its basic job or jobs not originally included in the definition of the system's requirements in changing environments. This can be conceptualized as actively minimizing the set of infeasible designs across different requirement scenarios \cite{Chalupnik2013}. Although this may be confused with other aspects of design changeability such as adaptability, the main commonality with the filtered outdegree definition is the ability of a design to be modified.

Other studies look at flexibility from a cost of change perspective which is important to realize the needed design changes. \citeauthor{Rapp2018} quantify design flexibility in terms of development and integration costs associated with adding a subsystem option to a set of design solutions \cite{Rapp2018}.

\citeauthor{Cardin2017} explore alternative flexible design scenarios by solving a multistage stochastic programming problem to minimize a cost function \cite{Cardin2017}. The authors solve a waste-to-energy system design problem to determine the appropriate upgrade capacities and times. The cost function for flexibility is formulated in terms of net present value of the system in previous work \cite{Cardin2016}. Several waste demand (requirements) scenarios are generated via Monte Carlo simulation. The average profit of all the scenarios for a given system design (upgrade levels and  times) is maximized subject to a number of constraints. The most important constraint governing the behavior of the decision maker is the nonanticipativity constraint. It implies that decisions made up to a certain time period during the lifecycle are made solely based on previous and current knowledge of the system demands (requirements). This approach simulates a realistic design problem that progresses over the course of a product's lifecycle. Furthermore, a set of solutions can be obtained simply by minimizing the cost function of each Monte Carlo sample to obtain a corresponding solution.

Flexibility can be considered in both the design and requirements spaces \cite{Ferguson2008}. When reviewing the available literature, it appears that quantifying changeability is performed largely in the requirements space rather than the design space and a methodology for mapping between the two spaces is required to identify the most flexible designs \cite{Tackett2014,Olewnik2004,Liu2008,Yannou2003}. Furthermore, when considering changeability due to changes in the operating environment and the product, it is important to consider a set of solutions that are changeable in order to leverage the added flexibility of the design solutions \cite{Olewnik2004,Liu2008,Suh2007}. A single point design that is flexible would not be justified if no alternatives are offered. Finally, among the mentioned aspects of changeability, scalability appears to be of relevance to remanufacturing since it involves upgrading the specifications of a product's components to achieve the required change. As a result, we will focus on set-based design principles while considering a metric for identifying scalable solution sets for remanufacturing purposes in Chapter \ref{ch:scalableSBD}.

Another important aspect of design changeability is robustness. Robustness characterizes a product's ability to be insensitive towards changing operational environments without the need for change or modification in contrast to flexibility. In the literature, robustness is usually associated with resilience, survivability, adaptability, and reliability depending on the design application being considered. 

For example, military products tend to incorporate resilience into their designs. Resilience has been defined as the capacity to cope with unanticipated dangers after they have become manifest; having the generic ability to cope with unforeseen challenges such as compromised performance during missions or changes to the mission objectives (requirements). This is an example where a resilient design can cope with changing operating environments and requirements \cite{Chalupnik2013,Small2019}. On the other hand, a robust design addresses changing requirements only \cite{Chalupnik2013}.

Survivability is used in the literature and draws a lot of parallels with the definition of resilience. It defines the ability of the design to cope with changing requirements (referred to as needs) and operational environments (referred to as operating context) \cite{McManus2007}.

Resilience is not explicitly defined by \citeauthor{Rehn2018} but appears to be a measure of the ability of a set-based solution to accommodate variabilities in the requirements \cite{Rehn2018}.

Reliability is defined in terms of the probability that the system will operate within or above the failure threshold for a nominal-is-better and a less-is-better objective function respectively \cite{Chalupnik2013}. 

Adaptability differs from robustness and resilience in the sense that the design modifies itself whilst in operation to accomplish its function in changing environments \cite{Chalupnik2013}. An example of this would be active spoilers on sports vehicles that dynamically adjust while the vehicle is in operation based on conditions presented.

\citeauthor{McManus2007} attempted to quantify robustness \cite{McManus2007}. In this study, changing needs and contexts are represented as discrete time periods, called epochs, during which the context and needs are stable. The tradespace for each epoch differs as the utility and cost of designs changes with changing requirements. Robustness is related to maintaining performance (capability) given changing operational environments (referred to as context) and requirements (referred to as needs). Value robustness is a special type of robustness related to value delivery which is the ability of the design to maintain Pareto efficiency on the tradespace across epochs.

Another study has quantified resilience and some of its aspects including survivability using a probability tree \cite{Small2019}. E.g., the reliability of an unmanned aerial vehicle is calculated by multiplying the reliability of the system during the mission with the probability that the system is available (availability). This is a measure of the ability of the system to perform reliably provided that the system is available to perform during the mission. The probability value for each resilience metric is computed based on design decisions and mission requirements. Monte Carlo simulation is used to investigate model uncertainty. Parameter uncertainty is then investigated in the tradespace by randomly sampling different parameter samples that drive the requirements and performances from a normal distribution \cite{Small2019}. 

{\color{red} A notable study introduces the concept of capability indices for use in robust design optimization \cite{Chen1999}. Design requirements are modelled using ranged sets while system performances are assumed to follow a uniform or normal distribution when subjected to noise variables (referred to as changing parameters in this thesis). The probability that the system performance will satisfy the required range is computed as the capability index and is the subject of a multi-objective optimization problem. A capability index of unity implies that the system's performance satisfies the requirement by a probability of 99.7\%. Any value greater than or equal to unity is considered satisfactory.}

In summary, design flexibility is associated with the ease of modifying the design to meet changing requirements outside the operational context. Robustness is associated with passively accommodating changing requirements outside the operational context. Resilience is an umbrella term for robustness and adaptability and describes the ability of the design to passively or actively cope with changing requirements and operational contexts. In this thesis, we will focus on quantifying and designing for flexibility and robustness as we are concerned with changes to requirements both during the product development cycle and its operational lifetime. 

% Simpson1997,Chen1996UsingDC use a metric to assess the capability of a family of designs, represented by a ranged set of top-level design specifications, to satisfy a ranged set of design requirements, obtained by solving the compromise DSP. Set-based approach using different scenarios obtained by setting different goal priorities. Uniform and Guassian distributions used to model system performances in relation to a ranged set of requirements. capability indices equal to or exceeding 1 imply that the system performance satisfies the ranged requirement by at least 99.7%. Capability indices maybe set as the constraint or the goal in a DSP depending on whether satisfying a range of design requirements is either a wish or a demand. Applications to design of passenger aircraft and solar powered irrigation systems are demonstrated.

% They model performances as response surfaces which deviate from their nominal values due to noise variables. A first order taylor expansion of the response surfaces is used to approximate the mean and standard deviation throughout the design space to calculate the relevant design capability indices for solving the DSP.

% Instead of approximating the distribution of the performance due to noise variables, we use a response surface that relates performance to the noise variables directly since performance functions do not necessarily follow a normal distribution as is assumed in this work.

% Similarly our reliability calculation is given by rational numbers greater than 0 with values greater than or equal to 1 implying full satisfaction of a given requirement PDF.

Since flexibility is usually defined in the tradespace via the filtered outdegree, we review some tradespace exploration studies in the context of flexible design in Section \ref{sec:tradespace}. In order to check robustness, the probability of meeting a requirement can be used to impose reliability constraints on the design solution under consideration. These design practices are showcased as part of our design framework for a product development problem where requirements change progressively in Chapter~\ref{ch:TSEcont}.

The studies reviewed in this section are summarized in Table \ref{table:changeabilitysummary} and classified based on the changeability aspects considered and the metrics used to quantify them. We also position our contributions in Chapters \ref{ch:scalableSBD} and \ref{ch:TSEcont} relative to some of the most notable studies that have been reviewed in this section.

Table \ref{table:changeabilitysummary} shows that our contributions focus entirely on external changes in requirements (such as changes in the operational context or the customer requirements). We focus specifically on scalability and robustness when considering changeability aspects. This is because in remanufacturing, a component is expected to adapt to increased requirement levels as opposed to new unforeseen requirements. Finally, both our contributions provide a method for determining the upgrade levels needed to accommodate the changing requirements by using design optimization.

The amount of design changeability embedded in a product is derived from its flexibility and robustness. Robustness is usually associated with design margins that are embedded in a design to absorb change. Metrics for quantifying these design margins are reviewed in the following section.

% \newcommand{\changeCW}{1.0cm}

% \begin{table}[h!]
% 	\centering
% 	\renewcommand{\arraystretch}{1.0}% Wider
% 	\footnotesize\addtolength{\tabcolsep}{-5pt}
% 	\caption{Summary of changeability aspects considered in the literature}
% 	\label{table:changeabilitysummary}
% 	\begin{tabular}{C{\changeCW}|C{\changeCW}C{\changeCW}|C{\changeCW}C{\changeCW}C{\changeCW}|C{\changeCW}C{\changeCW}C{\changeCW}|C{\changeCW}C{\changeCW}|C{\changeCW}}
% 	\hline\hline

% 	\multirow{3}{*}[-20.0pt]{\bf Contribution} & \multicolumn{2}{c|}{\bf Change type} & \multicolumn{3}{c|}{\bf Change effect} & \multicolumn{3}{c|}{\bf Metric} & \multicolumn{2}{c|}{\bf Upgrade levels} & \\ hline

% 	 & \rot{Internal} & \rotl{External} & \rot{Robust} & \rot{Modifiable} & \rotl{Scalable} & \rot{min cost} & \rot{Set size} & \rotl{\acs{FO}} & \rot{ Preset} & \rotl{Computed} & \rot{Set-based} \\
% 	 & & & & & & & & & & & \\ \hline
% 	%================================================================
% 	& 1 & 2 & 3 & 4 & 5 & 6 & 7 & 8 & 9 & 10 & 11 \\ 
% 	& 1 & 2 & 3 & 4 & 5 & 6 & 7 & 8 & 9 & 10 & 11 \\ 
% 	%================================================================
% 	\hline\hline
% 	\end{tabular}
% \end{table}

\newcommand{\changeCW}{0.55cm}
\newcommand{\mycontCW}{1.5cm}

\begin{table}[h!]
	\centering
	\renewcommand{\arraystretch}{1.0}% Wider
	\footnotesize\addtolength{\tabcolsep}{-5pt}
	\caption{Summary of changeability aspects considered in the literature}
	\label{table:changeabilitysummary}
	\begin{tabular}{lC{2.5cm}|C{\changeCW}C{\changeCW}C{\changeCW}C{\changeCW}C{\changeCW}C{\changeCW}C{\changeCW}C{\changeCW}C{\changeCW}C{\changeCW}C{\changeCW}|C{\mycontCW}C{\mycontCW}}
	\hline\hline

	\multicolumn{2}{c|}{ \multirow{2}{*}[-0.0pt]{\bf Feature} } & \multicolumn{12}{c}{\bf Contribution(s)} \\ \cline{3-15}
	 & & \cite{Tackett2014} & \cite{Olewnik2004} & \cite{Liu2008} & \cite{Suh2007} & \cite{Rehn2018} & \cite{Viscito2009} & \cite{Rapp2018} & $\dagger$ & \cite{McManus2007} & \cite{Small2019} & \cite{Chen1999} & Chapter \ref{ch:scalableSBD} & Chapter \ref{ch:TSEcont} \\ \hline
	%================================================================
	\multirow{2}{*}[-0.0pt]{\bf Change type} & Internal & ~ & ~ & \cmark & ~ & ~ & ~ & \cmark & ~ & ~ & \cmark & \cmark & ~ & ~ \\
	 & External & \cmark & \cmark & \cmark & \cmark & \cmark & \cmark & \cmark & \cmark & \cmark & \cmark & \cmark & \cmark & \cmark \\ \hline
	%================================================================
	\multirow{3}{*}[-0.0pt]{\bf Change effect} & Robust & ~ & \cmark & ~ & \cmark & ~ & ~ & \cmark & \cmark & \cmark & \cmark & \cmark & ~ & \cmark \\
	 & Modifiable & \cmark & ~ & ~ & ~ & ~ & ~ & \cmark & ~ & ~ & ~ & ~ & ~ & ~ \\
	 & Scalable & \cmark & \cmark & \cmark & \cmark & \cmark & \cmark & ~ & \cmark & \cmark & \cmark & ~ & \cmark & \cmark \\ 
	\hline\hline
	%================================================================
	\multirow{2}{*}[-0.0pt]{\bf Upgrade levels} & Predetermined & \cmark & ~ & ~ & \cmark & \cmark & \cmark & \cmark & ~ & \cmark & ~ & ~ & ~ & ~ \\
	 & Computed & ~ & \cmark & \cmark & ~ & ~ & ~ & ~ & \cmark & ~ & \cmark & \cmark & \cmark & \cmark \\ \hline	
	%================================================================
	\multirow{3}{*}[-0.0pt]{\bf Flexibility metric} & Cost function & \cmark & ~ & \cmark & \cmark & ~ & \cmark & \cmark & \cmark & \cmark & ~ & ~ & \cmark & \cmark \\
	 & Set size & ~ & \cmark & ~ & ~ & ~ & ~ & ~ & ~ & ~ & ~ & ~ & ~ & \cmark \\
	 & \acs{FO} & ~ & ~ & ~ & ~ & \cmark & \cmark & ~ & ~ & ~ & ~ & ~ & ~ & \cmark \\
	\hline\hline
	%================================================================
	\multirow{3}{*}[-0.0pt]{\bf Change type} & Environment Change & ~ & \cmark & ~ & ~ & ~ & \cmark & ~ & ~ & \cmark & \cmark & \cmark & \cmark & \cmark \\
	 & Requirement change & \cmark & \cmark & \cmark & \cmark & \cmark & \cmark & \cmark & \cmark & ~ & \cmark & \cmark & \cmark & \cmark \\ \hline
	%================================================================
	\multirow{3}{*}[-0.0pt]{\bf Robustness metric} & Reliability & ~ & ~ & ~ & ~ & ~ & ~ & ~ & ~ & ~ & ~ & \cmark & ~ & \cmark \\
	 & Capability & ~ & ~ & ~ & ~ & ~ & ~ & ~ & \cmark & \cmark & ~ & \cmark & ~ & ~ \\
	 & Probability chain & ~ & ~ & ~ & ~ & ~ & ~ & ~ & ~ & \cmark & \cmark & ~ & ~ & ~ \\
	\hline\hline
	%================================================================
	\multicolumn{2}{c|}{\bf Set-based} & ~ & \cmark & \cmark & \cmark & \cmark & ~ & \cmark & \cmark & ~ & \cmark & \cmark & \cmark & \cmark \\
	%================================================================
	\hline\hline
	\end{tabular}
	\\
	\footnote[2]{}\citeauthor{Cardin2017} \cite{Cardin2017}, \citeauthor{Cardin2016} \cite{Cardin2016}
\end{table}

%=============================== MARGINS ===============================%
\section{The use of design margins for managing uncertain requirements} 
\label{sec:margins}

Design margins accommodate changing requirements by providing a buffer before any change to the product is required. They are measured as a portion of a product's capability. Capability is defined as the set of possible values for a design parameter for which feasibility is maintained \cite{Eckert2019}.

Quantifying design margins involves measuring the constituents of margins: buffer and excess. Buffer is defined as the portion of a design's capability reserved for meeting variations in a requirement. Excess is the portion of a design's capability beyond the limits within which a requirement may vary \cite{Tackett2014}.

Design margins are incorporated into product design by augmenting the capability of a product to include parameter values beyond the initial ones that were intended to satisfy the requirements resulting in more excess. This can be referred to as overdesign \cite{Eckert2019}.

Design margins can be managed by quantifying them explicitly to assess the cost and risk of moving to a new design solution later in a product's lifecycle or development process.

Few studies in the literature have focused on quantifying buffer and excess for use as metrics in product design.

\citeauthor{Tackett2014} use the product system's capability of meeting design requirements to quantify the available excess capacity for evolving \cite{Tackett2014}. Evolvability depends on the product's capacity for upgrade which in turn depends on the available excess for upgrading the product's capabilities. Product capability is defined as a weighted sum of multiple excess values for each requirement that the product must meet. Requirements depend on changing parameters and are usually defined in the parameter space which encompasses all the possible parameter values. Excess values are obtained by finding the range of changing parameters that the product satisfies in addition to the product's design requirements.

\citeauthor{Tackett2014} present an interesting perspective for computing capability and excess, however, they assume independence of the requirements from one another \cite{Tackett2014}. The example used in Chapter \ref{ch:TSEcont} shows that design problems could feature requirements that are dependant on multiple changing parameters. A change in one of the changing parameters could impact several requirements simultaneously. As a result, a linear combination of the excess values obtained by calculating the range of the parameters satisfied by the design in excess of the requirements would overestimate their quantities.

Other studies define application specific capability and capacity measures \cite{Rehn2018,Cardin2017}. In the context of ship design, capability is defined as a function of the specifications of the equipment that the ship can be upgraded with. Capacity, on the other hand, is defined in terms of the available excess for transport and storage \cite{Rehn2018}. In the context of a waste-to-energy system, capability is defined in terms of the volume of the anaerobic digestion and gasifier tanks \cite{Cardin2017}. Such definitions are characterized by an interval bounded by the minimum and maximum carrying capacity (changing parameters) of the design and suffer from the same issues described above when using ranged sets to describe a design's capability and excess.

Design margins are also used as an umbrella term defined as the amount by which system specifications exceed requirements \cite{Cross2015}. \citeauthor{Cross2015} do not distinguish margins in terms of buffer and excess. The distribution of the changing parameters driving the requirements is known for a particular design problem defined at the current design stage. Buffer and excess are defined based on future changes to the distribution of the changing parameters such that the requirement spans a different range of values. However, for design purposes, \citeauthor{Cross2015} present a methodology for allocating design margins such that design performance is maximized via \ac{MDO} \cite{Cross2015}. So-called reliability levels are defined by the probability of satisfying the requirements must be maintained in the solution. This is one of the few studies that utilize design margins as a design metric that can be optimized. 

In another study, the performance of the design (the surface temperature of a thermal protection system) is subject to variabilities due to random fluctuations associated with the computational model used to calculate the temperature on the bottom surface of the system \cite{Villanueva2014}. Taking one sample from the calculated temperature distribution, capacity is defined as the maximum allowable temperature that may be reached by the system. Safety margins that are less than the capacity (maximum allowable temperature) are also defined and optimized in the study to minimize weight while maintaining a threshold probability of failure. In this study, there is only one requirement on the temperature of the bottom surface of the system. For such a problem, the description of capacity and safety margins is sufficient. 

{\color{red} Capability indices provide a useful tool for estimating reliability in the presence of changing parameters in multi-dimensional spaces. However, they do not provide a method for quantifying excess and buffer \cite{Chen1999}. The work can be extended to include such definitions by considering design capability indices greater than unity to be a form of overdesign. However, system performance is assumed to follow a normal distribution when subject to noise variables. This does not always hold when considering a variety of design problems.}

The need for scaling up the dimensionality of design problems in terms of number of requirements has been {\color{red} partially addressed by \citeauthor{Chen1999} \cite{Chen1999}. Majority of the reviewed studies use intervals to represent design margins and capabilities  which are not accurate representations of the sets of designs that satisfy them.} Finally, possible changes in the requirements should also be considered in order to incorporate the needed design flexibility. Methods for quantifying the degree of flexibility were reviewed in Section \ref{sec:changeability}.

We summarize the methods reviewed for computing design margins in Table \ref{table:marginssummary} in terms of the dimensionality of the requirements considered and the numerical method used for calculating them. They are also compared to the methods used as part of our framework in Chapter \ref{ch:TSEcont}.

Table \ref{table:marginssummary} shows that we present a novel method for quantifying and distinguishing the excess and buffer components of design margins. We use Monte Carlo integration to compute the size of the excess and buffer sets defined in a multi-dimensional parameter space.

A set of equally changeable solutions is needed to leverage the added design changeability. We review the most recent advances in \ac{SBD} in the following section to use them as part of our frameworks.

\renewcommand{\changeCW}{0.55cm}
\renewcommand{\mycontCW}{1.5cm}

\begin{table}[h!]
	\centering
	\renewcommand{\arraystretch}{1.0}% Wider
	\footnotesize\addtolength{\tabcolsep}{-5pt}
	\caption{Summary of design margin aspects considered in the literature}
	\label{table:marginssummary}
	\begin{tabular}{lC{2.5cm}|C{\changeCW}C{\changeCW}C{\changeCW}C{\changeCW}C{\changeCW}C{\changeCW}|C{\mycontCW}C{\mycontCW}}
	\hline\hline

	\multicolumn{2}{c|}{ \multirow{2}{*}[-0.0pt]{\bf Feature} } & \multicolumn{8}{c}{\bf Contribution(s)} \\ \cline{3-10}
	 & & \cite{Tackett2014} & \cite{Cansler2016} & \cite{Rehn2018} & \cite{Cross2015} & \cite{Villanueva2014} & \cite{Chen1999} & Chapter \ref{ch:scalableSBD} & Chapter \ref{ch:TSEcont} \\ \hline
	%================================================================
	\multirow{3}{*}[-0.0pt]{\bf Margins} & Excess & \cmark & \cmark & \cmark & ~ & ~ & ~ & ~ & \cmark \\
	 & Buffer & ~ & ~ & ~ & ~ & ~ & ~ & ~ & \cmark \\
	 & Excess + buffer & ~ & ~ & ~ & \cmark & \cmark & ~ & ~ & ~ \\ \hline
	%================================================================
	\multirow{2}{*}[-0.0pt]{\bf Calculation method} & Interval-based & ~ & \cmark & \cmark & \cmark & \cmark & ~ & ~ & ~ \\
	 & Integral-based & \cmark & ~ & ~ & ~ & ~ & \cmark & ~ & \cmark \\ \hline
	%================================================================
	\multicolumn{2}{c|}{\bf Used for design optimization} & ~ & ~ & ~ & \cmark & \cmark & \cmark & ~ & \cmark \\ \hline
	%================================================================
	\multicolumn{2}{c|}{\bf Multi-dimensional interactions} & ~ & ~ & ~ & ~ & ~ & \cmark & ~ & \cmark \\ \hline
	%================================================================
	\hline\hline
	\end{tabular}
\end{table}

%=========================== SET-BASED DESIGN ==========================%
\section{Set-based design principles and applications} 
\label{sec:SBD}

Due to the high level of uncertainty at the early phases of the product development process, designers have adopted iterative product design methods. Traditionally, the design problem is solved by selecting an initial design based on existing knowledge or expert opinion as an initial ``seed'' in the design space. The initial seed design is improved iteratively until a satisfactory design that meets the design requirements is reached. This paradigm is known as \ac{PBD} \cite{Qureshi2014, Kerga2014, SobekIi1999}. \ac{PBD} allows the design engineers to arrive at a solution in a short time frame. However, once the design engineers commit to a solution in the design phase, it becomes difficult to modify the design should the system requirements change during the later stages of the product development process \cite{Levandowski2014a, Carlson2000a}.

A possible remedy to the above shortcomings is to delay commitment to a single design early in the design stage. \Acf{SBD} is another design paradigm that addresses this by exploring alternative designs in the early stages of product development and delay commitment to a single design. The set of alternative designs is developed simultaneously until the variable parameters driving the requirements have been refined. Only the set of designs that has been refined by the updated requirements is developed further. This results in several designs rather than a single design that are gradually refined over the course of the product development process.

\citeauthor{SobekIi1999} identify three principles to be observed during \ac{SBD} \cite{SobekIi1999}. 1) The design space is explored to identify feasible designs comprising the \ac{FDS} with respect to each design requirement and quantify trade-offs between possible design solutions. 2) The intersection of the \acp{FDS} is identified in 1) while still maintaining flexibility in the offered design solutions. 3) The \ac{FDS} is gradually narrowed down by eliminating undesirable solutions as design requirements become more well-defined and constraints are tightened. It can be concluded that an \ac{SBD} methodology should feature (i) design maps of the \acp{FDS} that are transferable to ease communication between different engineering teams, (ii) must capture arbitrarily shaped \acp{FDS}, (iii) assess feasibility of design solutions efficiently to offset the longer lead time associated with \ac{SBD}, and (iv) have the ability to incorporate designer preferences as a means for eliminating designs.

There are several works that address the \ac{SBD} principles introduced by \citeauthor{SobekIi1999} quantitatively. They can be classified into works that focus on either design feasibility assessment or design space reduction based on performance and designer preferences.

Interval arithmetic has been used to map the \ac{FDS} \cite{Qureshi2014, Nahm2005}. \citeauthor{Qureshi2014} partition the design space into hyper-rectangle domains in which feasibility is assessed \cite{Qureshi2014}. If feasibility is not established throughout the hyper-rectangle, the domain is further subdivided and feasibility is checked in each subdivision until all feasible hyper-rectangles are identified. \textit{Noise} variables associated with uncertain parameters in the set-based context are quantified by means of intervals. Hyper-rectangles that lie within noise variable intervals are considered a subset of the robust design space. The method is intuitive and effective at reducing the design space to a manageable subspace. However, design spaces cannot always be captured by hyper-rectangles due to their irregular shapes. This is because uncertain parameters and design variables may affect several requirements simultaneously. This often causes the feasible regions that satisfy the requirements to assume highly irregular shapes including disconnected regions. Moreover, design requirements are often not given as analytic expressions of the design variables and parameters, but are obtained from simulation models. Fuzzy set theory has been used to accommodate design variable uncertainties in the context of \ac{SBD} \cite{Gventer1999}. However, fuzzy sets describe the membership of a quantity over an interval or a hyper-rectangle just like classical sets which may be inadequate for capturing arbitrarily shaped design spaces. The \ac{SBD} approach is similar to the notion of ranged sets described earlier \cite{Liu2008}.

Convex hulls have been used to identify the feasible sets while design constraints have been used to treat design requirements \cite{Kizer2014}. The constraints are perturbed to represent variability of the design requirements, resampling in proximity of the constraint is used to refine the convex hull and redefine the \ac{FDS}. The method can capture irregularly shaped design spaces and is intuitive. However, this methodology is computationally intensive due to the need for constant resampling as the design problem evolves (especially if expensive engineering models are used to calculate the constraints).

Another feasibility assessment tool is formulated using \aclp{BNC} (\acsp{BNC}) \cite{Shahan2012,Backlund2015,Rosen2015a}. The motivation of this work comes from using \ac{CP} to identify feasible solution sets \cite{Yannou2003}. \Ac{CP} requires analytical expressions of the system constraints to map the feasible regions. As mentioned above, such analytical expressions are not always available in simulation-based design problems. In these cases, metamodels can be used as surrogates of the constraint functions. \acp{BNC} use a set of training data generated by engineering models to train a \ac{KDE} for estimating a posterior probability distribution for feasible and infeasible design events. The decision surface is computed from the intersection of the two probability distributions and a threshold probability (typically 0.5) is used to render feasibility decisions \cite{Shahan2012}. The method is systematic and can accommodate a constant stream of data from the designer. Furthermore, the \ac{BNC} approach can render decision boundaries for irregularly shaped design spaces and produces a \ac{KDE} that can be communicated with other design teams for visualizing the feasible set of each subsystem in question. Finally, Monte Carlo simulation can be used to calculate the volume of the reduced design space for comparative studies \cite{Yannou2003}.

\citeauthor{Ge2005} introduced a surrogate-based \ac{SBD} methodology to facilitate interactive negotiations in design engineering groups who are responsible for design tasks at different hierarchical levels, i.e., at the system, subsystem, and component levels \cite{Ge2005}. Surrogate models are used to capture the interactions and the dynamics of the engineering systems and subsystems and are used to map \aclp{FDR} (\acsp{FDR}) and \aclp{EDR} (\acsp{EDR}) to satisfy design requirements and performances respectively. Finally, robustness is evaluated using the hypervolume of the \acp{EDR} as a metric. The larger the hypervolume, the more robust is the \ac{EDR} \cite{Taguchi1987}.

\ac{SBD} principles have also been extended to platform development \cite{Landahl2016,Suh2007} and conceptual design \cite{Jiachuan2003}. Platform assessment processes have been used to ensure feasibility of the narrowed-down set-based solutions in platform development of product variants. The process blocks are integrated in a \ac{PLM} architecture to facilitate information exchange between the platform assessment blocks \cite{Landahl2016}. \citeauthor{Jiachuan2003} employ a design synthesis technique to generate concepts using an agent-based modeling approach to conceptual design \cite{Jiachuan2003}. The generated concepts embody the \ac{FDS} for conceptual design.

So far, the studies reviewed provide means for identifying set-based solutions. \ac{SBD} principles involve narrowing down the \ac{FDS} to a handful of acceptable designs for further investigation and detailed design. Several works have presented a design or concept elimination methodology for narrowing down feasible sets by eliminating undesirable designs in terms of performance or designer preferences.

A diversity metric can be used to develop a representative cost for configurations within the \ac{FDS} of the design space associated with the risk of violating feasibility \cite{Doerry2014}. \citeauthor{Malak2009} use utility theory to make set-based decisions. Interval dominance criterion was used to eliminate designs when there is no overlap in their uncertainty ranges \cite{Malak2009}. The maximality criterion was used to make decisions involving design variables with overlapping uncertainty intervals. \citeauthor{Nahm2005} accommodate designer preferences in the form of ``preference numbers'' and functions \cite{Nahm2005}. The designer's preference structure spans design variables and requirements which may be a product of multidisciplinary analyses. However, as with fuzzy sets, the approach may not span arbitrarily shaped design spaces.

In most design problems, a number of competing objectives or attributes often arise. This is the case with \ac{SBD} problems. \citeauthor{Jiachuan2003} used a genetic algorithm to evaluate alternative design trade-offs in a component-based system synthesis problem \cite{Jiachuan2003}. A generalized weighted aggregate of fuzzy-set preferences was used as an optimization objective. \citeauthor{Avigad2009b} solved a trade-off problem based on the \ac{OAV} of each conceptual design in the design space \cite{Avigad2009b}. The two metrics are extracted from the Pareto sets associated with each set-based concept \cite{Avigad2009b}. Trade-off rules are subjective to designer preferences and are a good approach for accommodating designer preferences during design elimination.

\citeauthor{Miller2018} investigated a multi-fidelity approach to \ac{SBD}, where increasing levels of fidelity are concurrently met with increasing level of detail in the set-based solutions \cite{Miller2018}. The refinement is carried out over a modelling sequence to minimize the cost associated with modelling effort. Interval dominance is used to gradually narrow down the solution set for each model used.

\citeauthor{Hannapel2014} present an \ac{MDO} approach to set-based design by treating the design space boundaries as design variables for the system-level optimization problem \cite{Hannapel2014}. The discipline problems are solved individually for a specific objective and are coordinated by the system-level optimization problem. The objective of the system-level problem is an aggregate of design space hypervolume, weighted sum of individual discipline objectives and relaxable constraint violation. By solving the \ac{MDO} problem, the design space is narrowed down. Design performance is accommodated in the discipline-level optimization problems. The methodology assumed a design space in the form of a hyper-rectangle as prescribed by the design variable intervals. However, practical engineering design problems feature irregularly shaped design spaces \cite{Shahan2012}. Furthermore, the utilized weighted method assumes a convex attainable set for the objectives considered in the system-level optimization problem, which is not necessarily the case \cite{WardAthan1996}.

We classify the \ac{SBD} methods based on how sets of design solutions are represented. The two distinct representations that emerge are ranged sets \cite{Qureshi2014,Nahm2005,Olewnik2004,Liu2008,Suh2007} and response surfaces \cite{Kizer2014,Shahan2012,Yannou2003,Ge2005}. Response surfaces have the advantage of being able to capture irregularly shaped design spaces and are more conservative in comparison to hyper-rectangular sets. \citeauthor{Shahan2012} and \citeauthor{Yannou2003} accommodate nonlinear design requirements through various metamodels such as \acp{BNC} and \acp{PRS} used as surrogates of feasibility models \cite{Shahan2012,Yannou2003}.

\ac{SBD} methods consider predominantly computational design engineering problems. However, the surrogate models used by \citeauthor{Shahan2012} and \citeauthor{Yannou2003} can be used with experimental, testing, and operational data from the component being remanufactured \cite{Shahan2012,Yannou2003}. This makes surrogate models useful for a wide range of engineering design problems.

Finally, a number of techniques for narrowing down the set-based solution has been proposed. These techniques include Pareto set membership \cite{Olewnik2004,Miller2018}, optimality of a ranged set with respect to an objective function (design performance or flexibility) \cite{Hannapel2014,Liu2008,Suh2007}, and interval dominance for ranged sets \cite{Malak2009,Miller2018}.

A summary of these findings is given in Table \ref{table:SBDsummary} with respect to the principles of set-based design established by \citeauthor{SobekIi1999} \cite{SobekIi1999}. Our contributions in Chapters \ref{ch:scalableSBD} and \ref{ch:TSEcont} are also compared against the most recent advances in \ac{SBD}.

Table \ref{table:SBDsummary} shows that we address the limitation of using ranged sets to represent the solution set. We use response surfaces to identify set-based solutions in Chapter~\ref{ch:scalableSBD}. In Chapter~\ref{ch:TSEcont} we used a convex hull to represent our set-based solutions due to the finiteness and discreteness of the design space being considered. This is because the application example used in that chapter featured categorical and discrete design variables. However, as explained in Chapter~\ref{ch:conclusion} response surfaces should be used when adapting the methodology in Chapter~\ref{ch:TSEcont} for continuous or mixed variable problems.

\renewcommand{\changeCW}{0.55cm}
\renewcommand{\mycontCW}{1.2cm}

\begin{table}[h!]
	\centering
	\renewcommand{\arraystretch}{1.0}% Wider
	\footnotesize\addtolength{\tabcolsep}{-5pt}
	\caption{Summary of set-based approaches considered in the literature}
	\label{table:SBDsummary}
	\begin{tabular}{p{2.0cm}C{3cm}|C{\changeCW}C{\changeCW}C{\changeCW}C{\changeCW}C{\changeCW}C{\changeCW}C{\changeCW}C{\changeCW}C{\changeCW}C{\changeCW}C{\changeCW}|C{\mycontCW}C{\mycontCW}}
	\hline\hline

	\multicolumn{2}{c|}{ \multirow{2}{*}[-0.0pt]{\bf Features of reported sets} } & \multicolumn{13}{c}{\bf Contribution(s)} \\ \cline{3-15}
	 & & \cite{Qureshi2014} & \cite{Nahm2005} & \cite{Olewnik2004} & \cite{Liu2008} & \cite{Gventer1999} & \cite{Kizer2014} & $\dagger$ & \cite{Miller2018} & \cite{Hannapel2014} & \cite{Suh2007} & \cite{Malak2009} & Chapter \ref{ch:scalableSBD} & Chapter \ref{ch:TSEcont} \\ \hline
	%================================================================
	\multirow{4}{*}[-0.0pt]{\bf Set representation} & Interval & \cmark & ~ & \cmark & \cmark & ~ & ~ & ~ & \cmark & \cmark & \cmark & \cmark & ~ & ~ \\
	 & Fuzzy set & ~ & \cmark & ~ & ~ & \cmark & ~ & ~ & ~ & ~ & ~ & ~ & ~ & ~ \\
	 & Convex hull & ~ & ~ & ~ & ~ & ~ & \cmark & ~ & ~ & ~ & ~ & ~ & ~ & \cmark \\
	 & Metamodel & ~ & ~ & ~ & ~ & \cmark & ~ & \cmark & ~ & ~ & ~ & ~ & \cmark & ~ \\ \hline
	%================================================================
	\multirow{2}{*}[-0.0pt]{\bf Update ease} & Resampling & \cmark & \cmark & \cmark & \cmark & ~ & \cmark & ~ & \cmark & \cmark & \cmark & \cmark & ~ & \cmark \\
	 & No resampling & ~ & ~ & ~ & ~ & \cmark & ~ & \cmark & ~ & ~ & ~ & ~ & \cmark & ~ \\ 
	 \hline\hline
	%================================================================
	 & Pareto optimality & ~ & ~ & \cmark & ~ & ~ & ~ & ~ & ~ & ~ & ~ & ~ & ~ & \cmark \\
	 {\bf Reduction} & \ac{MDO} & ~ & ~ & ~ & \cmark & ~ & ~ & ~ & ~ & \cmark & \cmark & ~ & \cmark & \cmark \\ 
	 {\bf method} & Interval dominance & ~ & ~ & ~ & ~ & ~ & ~ & ~ & \cmark & ~ & ~ & \cmark & ~ & ~ \\ 
	 & Flexibility criteria & ~ & ~ & ~ & \cmark & ~ & ~ & ~ & ~ & ~ & \cmark & ~ & \cmark & \cmark \\ \hline
	%================================================================
	\multirow{2}{*}[-0.0pt]{\bf Shape} & Hyper-rectangle & \cmark & \cmark & \cmark & \cmark & \cmark & ~ & ~ & \cmark & \cmark & \cmark & \cmark & ~ & ~ \\
	& Arbitrary & ~ & ~ & ~ & ~ & ~ & \cmark & \cmark & ~ & ~ & ~ & ~ & \cmark & \cmark \\
   %================================================================
	\hline\hline
	\end{tabular}
	\footnote[2]{}\citeauthor{Shahan2012} \cite{Shahan2012}, \citeauthor{Yannou2003} \cite{Yannou2003}, \citeauthor{Ge2005} \cite{Ge2005}
\end{table}

Studies that utilize tradespace exploration will be reviewed in order to quantitatively compare sets of design solutions in Chapter \ref{ch:TSEcont}. 

%============================== TRADESPACE =============================%
\section{The use of tradespace exploration for quantifying design margins and flexibility} 
\label{sec:tradespace}

Several studies reviewed thus far have incorporated tradespace exploration strategies to quantify their design metrics and visualize their design solutions \cite{Rehn2018,McManus2007,Viscito2009,Small2019}. Other studies generate Pareto-optimal solutions exclusively in their analyses which is a form of tradespace exploration that focuses on designs that dominate the tradespace in terms of utility and cost \cite{Villanueva2014,Cross2015}.

\citeauthor{Viscito2009} and \citeauthor{Rehn2018} considered quantifying flexibility in terms of the filtered outdegree \cite{Viscito2009,Rehn2018}. \citeauthor{McManus2007} quantify survivability as a multi-attribute function for the tradespace \cite{McManus2007}. Another study similarly uses survivability among other aspects of resilience as the utility metric for tradespace exploration \cite{Small2019}.

Most studies focus on quantifying a robustness or resilience metric for use as a utility during tradespace exploration. This tends to lead to costly overdesigns in some cases, especially if solution sets are derived from the Pareto front on the tradespace. On the other hand, maximizing design flexibility derived from the filtered outdegree could lead to designs that do not meet requirement thresholds. It is worth investigating designs in the tradespace that are dominated by the Pareto front when considering robustness or reliability as a utility to avoid overdesign.

Furthermore, the design margin components reviewed in Section~\ref{sec:margins} such as buffer and excess could be used as part of the multi-attribute function governing our tradespace exploration strategy since they impact robustness.

The main utility of tradespace exploration lies in its ability to visualize and categorize designs into sets of solutions. \citeauthor{Viscito2009} use Pareto optimality to extract flexible cost-efficient designs from the tradespace and plot a reduced tradespace to further analyze this reduced set \cite{Viscito2009} . \citeauthor{Small2019} categorize solutions into sets by allocating design solutions into bins based on the value of the decision variable governing each solution \cite{Small2019}. Combining tradespace exploration with a robustness metric as the utility while investigating flexibility via filtered outdegree should prove useful for helping designers understand the tradeoff between these design metrics.

The gaps in the studies reviewed so far will be identified in order to position our contributions relative to the literature.

%=============================== SUMMARY ===============================%
\section{Research gaps and opportunities}
\label{sec:bgsummary}

The literature reviewed on this subject featured several studies that have attempted to quantify the aspects of changeability related to modifiability, robustness, and scalability of the design. We narrow down these studies to those that report set-based solutions. Among those studies, several research gaps in the field are evident.

\begin{itemize}
	\item None of the set-based representations featured a metamodel to represent the arbitrarily\-shaped solutions sets. Intervals where used for reporting set-based solutions.
	\item Only one study used \ac{MDO} to reduce the feasible design space to a flexible design set by narrowing down the intervals on design variables \cite{Hannapel2014}.
	\item Studies that used metamodels for reporting set-based solutions focused on identifying the feasible design space without further reduction \cite{Shahan2012,Yannou2003}.
\end{itemize}

We aim to address these research gaps in our first contribution. Our objective is to generate a set of changeable component designs that can be upgraded as necessary through remanufacturing to meet changing requirements that may arise at a product or system's \ac{EOL}. 

We propose a systematic design space reduction methodology using optimization and response surfaces. An optimization problem is formulated to include parameters reflecting the requirements at the current stage of the product development process or its lifecycle. We then obtain a set of parametric optimal solutions to maximize the performance of the remanufactured products. 

Since designers must commit to a solution eventually, it is important to consider the {changeability} of a product throughout its lifecycle and not just at the current stage to allow products to retain most of their economic value \cite{Fricke2005}. We therefore incorporate a scalability constraint in our \ac{SBD} methodology to further reduce our solutions set to readily changeable designs. The scalability constraint is evaluated in the requirements space since changeability and scalability by proxy are defined in the requirements space. The set of possible parameter values related to performance requirements are used as a proxy for the requirements space. We provide a mapping between the design space and the parameter space to transfer scalable designs identified in the parameter space back to the design space.

The proposed methodology is based on:
\begin{itemize}
	\item surrogates of the computational models for rapid evaluations during optimization,
	\item numerical optimization for identifying the best performing feasible designs for different design parameter values, i.e., a set of \textit{parametric} optimal designs,%. This is the first set-based solution.
	\item response surfaces of the parametric optimal design solutions that provide a mapping of design solutions between the design and parameter spaces,
	\item sensitivity analysis of design variables with respect to design parameters,
	\item a remanufacturing constraint based on the sensitivity of design variables to design parameters and manufacturing process capabilities that reduces the set of parametric optimal designs to a set of {scalable} optimal designs in the parameter space.
\end{itemize}

%=======================================================================%
%=======================================================================%

The reviewed literature suggests that the study of excess is relatively nascent. A methodology on where to place excess strategically in the design is required. The successful implementation of such a methodology requires a metric for quantifying  design excess while providing a means to manage the change in requirements likely to be encountered throughout a product's lifetime or development process. Finally, the value of excess must be estimated and traded against during system design \cite{Long2017}. In this field, the following research gaps have been identified.

\begin{itemize}
	\item There appears to be no attempt to quantify design margins (or its constituents: buffer and excess) in a multi-dimensional space by means other than intervals. 
	\item Although there is an abundance of studies on quantifying design robustness and flexibility, the tradeoff between these important aspects of design for changing requirements has not been investigated within a tradespace exploration framework. 
	\item Little work has been done to examine the evolutionary path of products in response to changes in their environment or requirements hence the need for an epoch-based analysis that explores product changes in a progressive manner as requirements change \cite{Long2017,Cardin2017}.
\end{itemize}

In our second contribution, we propose a methodology where we compute design reliability, capability, buffer, and excess  in a multi-dimensional parameter space. We use Monte Carlo simulation to chain multiple requirements (given by their \acp{PDF}) together to generate a requirement arc across multiple epochs. This formulation of the requirements captures the progressive nature of a product's development process and lifetime. A corresponding design arc is found by optimization such that excess is minimized while reliability is maintained above a threshold. In this manner, Monte Carlo simulation is used to generate multiple requirement arcs to obtain a set of solutions that balances robustness with flexibility by minimizing excess.

Our approach differs from the one in \cite{Cardin2017} in that we obtain set-based solutions by minimizing a cost function in terms of excess. Furthermore, categorical design variables are considered during the progressive upgrade of the design. 
%
Our methodology features the following elements.
%BulletList
\begin{itemize}
	\item We determine design capability relative to a feasibility constraint in the parameter space using Monte Carlo simulation.
	\item We model requirements using \acp{PDF}.
	\item We compute buffer and excess relative to requirements in the parameter space using Monte Carlo simulation.
	\item We use epochs for chaining multiple requirement \acp{PDF} to generate requirement arcs.
	\item We formulate and solve an optimization problem \cite{Rapp2018} for determining the design arc that minimizes excess while satisfying reliability constraints for a corresponding requirement arc.
	\item We use Monte Carlo simulation to generate a set of optimal design arcs with respect to excess.
	\item We generate a tradespace for visualizing the set of optimal design arcs and positioning it relative to sets that maximize filtered outdegree (flexible design set) and sets that satisfy the largest number of requirement arcs without regards to minimizing excess (robust design sets).
\end{itemize}

{\color{red} 
%======================= OPTIMIZATION ALGORITHMS =======================%
\section{Comparison of optimization algorithms for model-based design optimization} 
\label{sec:optLR}

We review several notable optimization algorithms used in the design literature to select an appropriate algorithm suited for the application example in this thesis. We seek an algorithm for optimizing problems involving blackbox models. Blackbox models are defined as any process that when provided an input, returns an output with no analytical knowledge of the inner workings of the blackbox \cite{Audet2017}. This description holds for simulation-based engineering models such as those presented in this thesis. This is because such models are not necessarily smooth, continuous or differentiable. Furthermore, such models are expensive to evaluate. Finally, derivatives of such blackbox models are difficult to estimate or compute. 

\Acfp{GA} are a popular choice when blackbox optimization problems are involved since they do not require derivative information. However, convergence of \acp{GA} is highly dependant on tuning the hyperparameters, encoding, and crossover techniques for the problem considered \cite{Audet2017}. We attempt to qualify our methods for a wide variety of design problems and do not consider such algorithms as a result.

\Ac{NM} is another popular heuristic method based on simplices. Convergence proof for such algorithms does not provide much insight on when the algorithm converges to a solution. In practice, \ac{NM} is effective, as it converges to a solution when the problem is nicely behaved. However, counter examples such as the McKinnon example show that \ac{NM} may converge to a non-minimizer for a convex function \cite{Audet2017}. As a result, such heuristic algorithms are not considered for the examples in this thesis.

\Ac{MADS} is a direct-search algorithm with rigorous convergence properties. It is based on advancement of the \ac{CS} and \ac{GPS} algorithms and features an optional search step and a poll step. The search step allows the algorithm to apply heuristics to break free of local minimizers while the poll step performs a localized search around the incumbent solution. This algorithm is shown to converge for both constrained and unconstrained blackbox optimization problems. This advancement in direct search strategies is enabled by the use a richer set of polling directions as defined by two distinct parameters that control the mesh and frame size in which polling is done. As with all direct search methods, \ac{MADS} does not require derivative information.

The previous direct search methods can be used with surrogate models of the expensive simulation model. In such frameworks a list of trail points such as those generated during the poll step of \ac{MADS} are evaluated using the surrogate to rank them in terms of decreasing the objective function. The true expensive model is then exercised at trail points in the same order opportunistically to avoid evaluating trail points that will be discarded by the algorithm. The surrogate model can be updated dynamically by the algorithm to improve its accuracy around the incumbent solution. The use of such methods is strongly recommended when surrogates for the expensive model are available. This is the case with examples used in the thesis. However, such methods still involve evaluations of the expensive blackbox. This would result in an immense computational cost when executing the set-based approaches in this thesis by solving many parametric optimization problems. Instead, we use a static surrogate model based on an ensemble of response surfaces to obtain the best possible estimates of the objective and constraints functions. Since the derivatives of this ensemble of surrogates are also difficult to estimate, we resort to \ac{MADS} for obtaining solutions to the optimization problems in this thesis.

We investigate \ac{RBDO} methods since the subject of this thesis is to design products for changing requirements. In the application example, a reliability constraint must be satisfied subject to a changing requirement defined by a random joint \ac{PDF}. \Ac{MPP} methods are used for solving problems where the objective and constraints are functions of random variables. However, \ac{MPP} methods can suffer in terms of accuracy when estimating reliability for constraints that are highly nonlinear \cite{Du2010}. \Ac{MCS} is shown to provide good accuracy for estimating reliability when used with \ac{LH} sampling \cite{Lehky2018}. This allows \ac{MCS} to scale proportionally with the dimensionality of the problem.

The application example in this thesis involves optimization problems with mixed variables which lend themselves to multistage stochastic programming. In the literature, Lagrange relaxation methods are used for decomposing an integer programming problem into several scenarios corresponding to the different requirements that can occur. Constraints such as nonanticipativity are penalized with Lagrange multipliers in the objective function. Since the variables considered are discrete, a duality gap exists and the search algorithm is terminated when the gap is lower than a certain value. This approach is useful for determining a single series of decisions that are insensitive to the different requirement scenarios that can arise. To leverage the added flexibility of our designs, we aim to solve several such problems for many different scenarios to obtain a set of solutions. Since our reliability constraints are expensive to compute, we resort to a mixed variable extension of \ac{MADS} that is suited for such problems \cite{Abramson2009}. It is worth noting that heuristic methods can solve mixed variable programming problems. However, they suffer from the same drawbacks when used for continous variables due to their obscure convergence properties.

The algorithms in this thesis are developed such that different optimization algorithms can be used interchangeably depending on the problem being solved.}

In the following chapter, we describe an application example from the industry to demonstrate the efficacy of our frameworks. In this example, we consider the remanufacturing of an aeroengine structural component that is subject to changing load requirements. The details of our contributions are outlined in the following chapters.