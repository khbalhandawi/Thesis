%%%%%%%%%%%%%%%%%%%%%%%%%%%%%%%%%%%%%%%%%%%%%%%%%%%%%%%
%%                     Conclusion                    %%
%%%%%%%%%%%%%%%%%%%%%%%%%%%%%%%%%%%%%%%%%%%%%%%%%%%%%%%
\chapter{Conclusions and Future Work}
\chaptermark{Conclusions and Future Work}
\label{ch:conclusion}
%%%%%%%%%%%%%%%%%%%%%%%%%%%%%%%%%%%%%%%%%%%%%%%%%%%%%%%

The thesis presented a novel design methodology for managing problems involving uncertain requirements and parameters. The developed frameworks were implemented by the use of parametric design optimization to exhaustively explore multi-dimensional design and parameter spaces and is distinguished by the following features.

\begin{itemize}
    \item It manages an instantaneous or gradual change in design requirements.
    \item It constructs surrogate models of expensive engineering models to mitigate the computational cost of exploring the design and parameter spaces.
    \item It utilizes rigorous derivative free optimization tools for exploring multi-dimensional design and parameter spaces without the need for gradients or their approximation.
    \item It provides sets of design solutions represented by response surfaces or convex hulls for multi-dimensional continuous or discrete design spaces respectively.
    \item Provides a mathematical formulation for qualitative design characteristics such as flexibility, scalability and excess.
\end{itemize}

A total of four algorithms were developed, demonstrated and applied to multiple remanufacturing design problems of a \ac{TRS} with different design variables, design space types (continuous or categorical) and varying requirements change scenarios (instantaneous or gradual). 

A method for mapping between design and parameter spaces using response surfaces was provided in Algorithm~\ref{algo:PODalgo}. A scalability constraint was formulated and applied to a remanufacturing design problem of a \ac{TRS} in Algorithm~\ref{algo:SODalgo}. These two algorithms are combined to identify a set of scalable optimal designs in the parameter space of a design problem when an instantaneous change in requirements occurs. Application of these algorithms to the \ac{TRS} remanufacturing problem provided a relatively small albeit varied set of scalable design solutions. This set of solutions is useful for reducing the design space to a manageable set of solutions for further development and detailed analysis.

Two additional algorithms were developed to manage a cascade of requirement changes at discrete times in a product's lifecycle or development cycle. As part of Algorithm~\ref{algo:SBDOptalgo}, three design metrics were quantified mathematically. They allow designers to quantify the capability, buffer and excess embedded in a design. The mathematical formulation of these design metrics is novel and allows them to be quantified in a multi-dimensional parameter space where multiple parameters may change simultaneously. The algorithm favors designs that minimize cost or excess throughout the product's cycle. This algorithm was applied to a \ac{TRS} remanufacturing problem featuring a categorical design space governed by a finite set of possible design choices. The resulting set-based solution obtained by this algorithm was displayed on a tradespace.

The fourth and final algorithm, Algorithm~\ref{algo:SBDRobustalgo} was used to obtain robust and flexible design sets using tools and definitions from the literature that were adapted for a gradual change in requirements. The algorithm was also applied to the categorical \ac{TRS} remanufacturing problem and its resulting sets were superimposed on the same tradespace used to visualize the set-based solution obtained by minimizing excess. The tradespace revealed that utilizing design optimization can help balance the amount of robustness and flexibility in a design, a problem which is difficult to address in the early stages of the product cycle due to the relatively high uncertainty during this time period.

Finally, we proposed some stochastic optimization tools that can be used to substitute parts of Algorithms~\ref{algo:PODalgo} and \ref{algo:SBDOptalgo} to make managing design and parameters spaces with relatively large dimensionality more tractable.

To the best of our knowledge, Algorithms~\ref{algo:SODalgo} and \ref{algo:SBDOptalgo} and their metrics and constraints are novel and are a first attempt at a systematic approach for managing a wide range of requirement change scenarios that are frequently encountered in the industry.

\section{Recommendations for Future Work}
\label{sec:futurework}

In this thesis we have developed novel systematic design approaches for managing a wide range of requirement change scenarios. However, there is considerable room for improvement. In this thesis we considered continuous or categorical design spaces independently. A framework for solving problems involving mixed variable design spaces is needed to identify the most efficient design combinations possible. 

Although we have provided a varied set of tools for quantifying design changeability, the cost of change must be considered when implementing the change in order to satisfy the definition of changeability \cite{Lawand2019}.

Although we consider the cumulative weight as a proxy for lifecycle cost, this definition is not sufficient and more advanced lifecycle modelling tools should be used for computing cost \cite{Lawand2019}. This relates directly to the lifecycle of the component and its importance to circular economy principles. A changeability cost threshold must be observed by designers when making remanufacturing decisions \cite{Ross2008}. 

Future studies could also focus on other product recovery activities along the innermost circles of a \ac{CE} such as repair and reuse as they may preserve valuable resources more efficiently than remanufacturing. Nevertheless, these options need not be mutually exclusive.

The \acp{PDF} used to model the uncertain requirements in this thesis assume that the parameters governing them are not correlated. Such correlations can be easily modelled using a covariance matrix as part of the the requirement formulation to involve more general scenarios. 

Importance sampling was used to make the management of a large number of requirement change scenarios more tractable. More advanced sampling techniques can be used to make the parametric studies used in this thesis more efficient. One such approach was demonstrated in the previous chapter.

By taking advantage of advances in numerical derivative free optimization and methods for quantifying lifecycle costs, a huge variety of requirement change scenarios can be managed in design optimization. Building on the methods developed in this thesis, should allow designers to make decisions in the early stages of the product's developement process or lifecycle when innovation potential is at its highest.